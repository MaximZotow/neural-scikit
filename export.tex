\documentclass[11pt]{article}

    \usepackage[breakable]{tcolorbox}
    \usepackage{parskip} % Stop auto-indenting (to mimic markdown behaviour)
    
    \usepackage{iftex}
    \ifPDFTeX
    	\usepackage[T1]{fontenc}
    	\usepackage{mathpazo}
    \else
    	\usepackage{fontspec}
    	\setmainfont{Times New Roman}
    	\setmonofont{JetBrains Mono}
    	\setsansfont{JetBrains Mono}
    \fi

    % Basic figure setup, for now with no caption control since it's done
    % automatically by Pandoc (which extracts ![](path) syntax from Markdown).
    \usepackage{graphicx}
    % Maintain compatibility with old templates. Remove in nbconvert 6.0
    \let\Oldincludegraphics\includegraphics
    % Ensure that by default, figures have no caption (until we provide a
    % proper Figure object with a Caption API and a way to capture that
    % in the conversion process - todo).
    \usepackage{caption}
    \DeclareCaptionFormat{nocaption}{}
    \captionsetup{format=nocaption,aboveskip=0pt,belowskip=0pt}

    \usepackage{float}
    \floatplacement{figure}{H} % forces figures to be placed at the correct location
    \usepackage{xcolor} % Allow colors to be defined
    \usepackage{enumerate} % Needed for markdown enumerations to work
    \usepackage{geometry} % Used to adjust the document margins
    \usepackage{amsmath} % Equations
    \usepackage{amssymb} % Equations
    \usepackage{textcomp} % defines textquotesingle
    % Hack from http://tex.stackexchange.com/a/47451/13684:
    \AtBeginDocument{%
        \def\PYZsq{\textquotesingle}% Upright quotes in Pygmentized code
    }
    \usepackage{upquote} % Upright quotes for verbatim code
    \usepackage{eurosym} % defines \euro
    \usepackage[mathletters]{ucs} % Extended unicode (utf-8) support
    \usepackage{fancyvrb} % verbatim replacement that allows latex
    \usepackage{grffile} % extends the file name processing of package graphics 
                         % to support a larger range
    \makeatletter % fix for old versions of grffile with XeLaTeX
    \@ifpackagelater{grffile}{2019/11/01}
    {
      % Do nothing on new versions
    }
    {
      \def\Gread@@xetex#1{%
        \IfFileExists{"\Gin@base".bb}%
        {\Gread@eps{\Gin@base.bb}}%
        {\Gread@@xetex@aux#1}%
      }
    }
    \makeatother
    \usepackage[Export]{adjustbox} % Used to constrain images to a maximum size
    \adjustboxset{max size={0.9\linewidth}{0.9\paperheight}}

    % The hyperref package gives us a pdf with properly built
    % internal navigation ('pdf bookmarks' for the table of contents,
    % internal cross-reference links, web links for URLs, etc.)
    \usepackage{hyperref}
    % The default LaTeX title has an obnoxious amount of whitespace. By default,
    % titling removes some of it. It also provides customization options.
    \usepackage{titling}
    \usepackage{longtable} % longtable support required by pandoc >1.10
    \usepackage{booktabs}  % table support for pandoc > 1.12.2
    \usepackage[inline]{enumitem} % IRkernel/repr support (it uses the enumerate* environment)
    \usepackage[normalem]{ulem} % ulem is needed to support strikethroughs (\sout)
                                % normalem makes italics be italics, not underlines
    \usepackage{mathrsfs}
    

    
    % Colors for the hyperref package
    \definecolor{urlcolor}{rgb}{0,.145,.698}
    \definecolor{linkcolor}{rgb}{.71,0.21,0.01}
    \definecolor{citecolor}{rgb}{.12,.54,.11}

    % ANSI colors
    \definecolor{ansi-black}{HTML}{3E424D}
    \definecolor{ansi-black-intense}{HTML}{282C36}
    \definecolor{ansi-red}{HTML}{E75C58}
    \definecolor{ansi-red-intense}{HTML}{B22B31}
    \definecolor{ansi-green}{HTML}{00A250}
    \definecolor{ansi-green-intense}{HTML}{007427}
    \definecolor{ansi-yellow}{HTML}{DDB62B}
    \definecolor{ansi-yellow-intense}{HTML}{B27D12}
    \definecolor{ansi-blue}{HTML}{208FFB}
    \definecolor{ansi-blue-intense}{HTML}{0065CA}
    \definecolor{ansi-magenta}{HTML}{D160C4}
    \definecolor{ansi-magenta-intense}{HTML}{A03196}
    \definecolor{ansi-cyan}{HTML}{60C6C8}
    \definecolor{ansi-cyan-intense}{HTML}{258F8F}
    \definecolor{ansi-white}{HTML}{C5C1B4}
    \definecolor{ansi-white-intense}{HTML}{A1A6B2}
    \definecolor{ansi-default-inverse-fg}{HTML}{FFFFFF}
    \definecolor{ansi-default-inverse-bg}{HTML}{000000}

    % common color for the border for error outputs.
    \definecolor{outerrorbackground}{HTML}{FFDFDF}

    % commands and environments needed by pandoc snippets
    % extracted from the output of `pandoc -s`
    \providecommand{\tightlist}{%
      \setlength{\itemsep}{0pt}\setlength{\parskip}{0pt}}
    \DefineVerbatimEnvironment{Highlighting}{Verbatim}{commandchars=\\\{\}}
    % Add ',fontsize=\small' for more characters per line
    \newenvironment{Shaded}{}{}
    \newcommand{\KeywordTok}[1]{\textcolor[rgb]{0.00,0.44,0.13}{\textbf{{#1}}}}
    \newcommand{\DataTypeTok}[1]{\textcolor[rgb]{0.56,0.13,0.00}{{#1}}}
    \newcommand{\DecValTok}[1]{\textcolor[rgb]{0.25,0.63,0.44}{{#1}}}
    \newcommand{\BaseNTok}[1]{\textcolor[rgb]{0.25,0.63,0.44}{{#1}}}
    \newcommand{\FloatTok}[1]{\textcolor[rgb]{0.25,0.63,0.44}{{#1}}}
    \newcommand{\CharTok}[1]{\textcolor[rgb]{0.25,0.44,0.63}{{#1}}}
    \newcommand{\StringTok}[1]{\textcolor[rgb]{0.25,0.44,0.63}{{#1}}}
    \newcommand{\CommentTok}[1]{\textcolor[rgb]{0.38,0.63,0.69}{\textit{{#1}}}}
    \newcommand{\OtherTok}[1]{\textcolor[rgb]{0.00,0.44,0.13}{{#1}}}
    \newcommand{\AlertTok}[1]{\textcolor[rgb]{1.00,0.00,0.00}{\textbf{{#1}}}}
    \newcommand{\FunctionTok}[1]{\textcolor[rgb]{0.02,0.16,0.49}{{#1}}}
    \newcommand{\RegionMarkerTok}[1]{{#1}}
    \newcommand{\ErrorTok}[1]{\textcolor[rgb]{1.00,0.00,0.00}{\textbf{{#1}}}}
    \newcommand{\NormalTok}[1]{{#1}}
    
    % Additional commands for more recent versions of Pandoc
    \newcommand{\ConstantTok}[1]{\textcolor[rgb]{0.53,0.00,0.00}{{#1}}}
    \newcommand{\SpecialCharTok}[1]{\textcolor[rgb]{0.25,0.44,0.63}{{#1}}}
    \newcommand{\VerbatimStringTok}[1]{\textcolor[rgb]{0.25,0.44,0.63}{{#1}}}
    \newcommand{\SpecialStringTok}[1]{\textcolor[rgb]{0.73,0.40,0.53}{{#1}}}
    \newcommand{\ImportTok}[1]{{#1}}
    \newcommand{\DocumentationTok}[1]{\textcolor[rgb]{0.73,0.13,0.13}{\textit{{#1}}}}
    \newcommand{\AnnotationTok}[1]{\textcolor[rgb]{0.38,0.63,0.69}{\textbf{\textit{{#1}}}}}
    \newcommand{\CommentVarTok}[1]{\textcolor[rgb]{0.38,0.63,0.69}{\textbf{\textit{{#1}}}}}
    \newcommand{\VariableTok}[1]{\textcolor[rgb]{0.10,0.09,0.49}{{#1}}}
    \newcommand{\ControlFlowTok}[1]{\textcolor[rgb]{0.00,0.44,0.13}{\textbf{{#1}}}}
    \newcommand{\OperatorTok}[1]{\textcolor[rgb]{0.40,0.40,0.40}{{#1}}}
    \newcommand{\BuiltInTok}[1]{{#1}}
    \newcommand{\ExtensionTok}[1]{{#1}}
    \newcommand{\PreprocessorTok}[1]{\textcolor[rgb]{0.74,0.48,0.00}{{#1}}}
    \newcommand{\AttributeTok}[1]{\textcolor[rgb]{0.49,0.56,0.16}{{#1}}}
    \newcommand{\InformationTok}[1]{\textcolor[rgb]{0.38,0.63,0.69}{\textbf{\textit{{#1}}}}}
    \newcommand{\WarningTok}[1]{\textcolor[rgb]{0.38,0.63,0.69}{\textbf{\textit{{#1}}}}}
    
    
    % Define a nice break command that doesn't care if a line doesn't already
    % exist.
    \def\br{\hspace*{\fill} \\* }
    % Math Jax compatibility definitions
    \def\gt{>}
    \def\lt{<}
    \let\Oldtex\TeX
    \let\Oldlatex\LaTeX
    \renewcommand{\TeX}{\textrm{\Oldtex}}
    \renewcommand{\LaTeX}{\textrm{\Oldlatex}}
    % Document parameters
    % Document title
    \title{export}
    
    
    
    
    
% Pygments definitions
\makeatletter
\def\PY@reset{\let\PY@it=\relax \let\PY@bf=\relax%
    \let\PY@ul=\relax \let\PY@tc=\relax%
    \let\PY@bc=\relax \let\PY@ff=\relax}
\def\PY@tok#1{\csname PY@tok@#1\endcsname}
\def\PY@toks#1+{\ifx\relax#1\empty\else%
    \PY@tok{#1}\expandafter\PY@toks\fi}
\def\PY@do#1{\PY@bc{\PY@tc{\PY@ul{%
    \PY@it{\PY@bf{\PY@ff{#1}}}}}}}
\def\PY#1#2{\PY@reset\PY@toks#1+\relax+\PY@do{#2}}

\@namedef{PY@tok@w}{\def\PY@tc##1{\textcolor[rgb]{0.73,0.73,0.73}{##1}}}
\@namedef{PY@tok@c}{\let\PY@it=\textit\def\PY@tc##1{\textcolor[rgb]{0.24,0.48,0.48}{##1}}}
\@namedef{PY@tok@cp}{\def\PY@tc##1{\textcolor[rgb]{0.61,0.40,0.00}{##1}}}
\@namedef{PY@tok@k}{\let\PY@bf=\textbf\def\PY@tc##1{\textcolor[rgb]{0.00,0.50,0.00}{##1}}}
\@namedef{PY@tok@kp}{\def\PY@tc##1{\textcolor[rgb]{0.00,0.50,0.00}{##1}}}
\@namedef{PY@tok@kt}{\def\PY@tc##1{\textcolor[rgb]{0.69,0.00,0.25}{##1}}}
\@namedef{PY@tok@o}{\def\PY@tc##1{\textcolor[rgb]{0.40,0.40,0.40}{##1}}}
\@namedef{PY@tok@ow}{\let\PY@bf=\textbf\def\PY@tc##1{\textcolor[rgb]{0.67,0.13,1.00}{##1}}}
\@namedef{PY@tok@nb}{\def\PY@tc##1{\textcolor[rgb]{0.00,0.50,0.00}{##1}}}
\@namedef{PY@tok@nf}{\def\PY@tc##1{\textcolor[rgb]{0.00,0.00,1.00}{##1}}}
\@namedef{PY@tok@nc}{\let\PY@bf=\textbf\def\PY@tc##1{\textcolor[rgb]{0.00,0.00,1.00}{##1}}}
\@namedef{PY@tok@nn}{\let\PY@bf=\textbf\def\PY@tc##1{\textcolor[rgb]{0.00,0.00,1.00}{##1}}}
\@namedef{PY@tok@ne}{\let\PY@bf=\textbf\def\PY@tc##1{\textcolor[rgb]{0.80,0.25,0.22}{##1}}}
\@namedef{PY@tok@nv}{\def\PY@tc##1{\textcolor[rgb]{0.10,0.09,0.49}{##1}}}
\@namedef{PY@tok@no}{\def\PY@tc##1{\textcolor[rgb]{0.53,0.00,0.00}{##1}}}
\@namedef{PY@tok@nl}{\def\PY@tc##1{\textcolor[rgb]{0.46,0.46,0.00}{##1}}}
\@namedef{PY@tok@ni}{\let\PY@bf=\textbf\def\PY@tc##1{\textcolor[rgb]{0.44,0.44,0.44}{##1}}}
\@namedef{PY@tok@na}{\def\PY@tc##1{\textcolor[rgb]{0.41,0.47,0.13}{##1}}}
\@namedef{PY@tok@nt}{\let\PY@bf=\textbf\def\PY@tc##1{\textcolor[rgb]{0.00,0.50,0.00}{##1}}}
\@namedef{PY@tok@nd}{\def\PY@tc##1{\textcolor[rgb]{0.67,0.13,1.00}{##1}}}
\@namedef{PY@tok@s}{\def\PY@tc##1{\textcolor[rgb]{0.73,0.13,0.13}{##1}}}
\@namedef{PY@tok@sd}{\let\PY@it=\textit\def\PY@tc##1{\textcolor[rgb]{0.73,0.13,0.13}{##1}}}
\@namedef{PY@tok@si}{\let\PY@bf=\textbf\def\PY@tc##1{\textcolor[rgb]{0.64,0.35,0.47}{##1}}}
\@namedef{PY@tok@se}{\let\PY@bf=\textbf\def\PY@tc##1{\textcolor[rgb]{0.67,0.36,0.12}{##1}}}
\@namedef{PY@tok@sr}{\def\PY@tc##1{\textcolor[rgb]{0.64,0.35,0.47}{##1}}}
\@namedef{PY@tok@ss}{\def\PY@tc##1{\textcolor[rgb]{0.10,0.09,0.49}{##1}}}
\@namedef{PY@tok@sx}{\def\PY@tc##1{\textcolor[rgb]{0.00,0.50,0.00}{##1}}}
\@namedef{PY@tok@m}{\def\PY@tc##1{\textcolor[rgb]{0.40,0.40,0.40}{##1}}}
\@namedef{PY@tok@gh}{\let\PY@bf=\textbf\def\PY@tc##1{\textcolor[rgb]{0.00,0.00,0.50}{##1}}}
\@namedef{PY@tok@gu}{\let\PY@bf=\textbf\def\PY@tc##1{\textcolor[rgb]{0.50,0.00,0.50}{##1}}}
\@namedef{PY@tok@gd}{\def\PY@tc##1{\textcolor[rgb]{0.63,0.00,0.00}{##1}}}
\@namedef{PY@tok@gi}{\def\PY@tc##1{\textcolor[rgb]{0.00,0.52,0.00}{##1}}}
\@namedef{PY@tok@gr}{\def\PY@tc##1{\textcolor[rgb]{0.89,0.00,0.00}{##1}}}
\@namedef{PY@tok@ge}{\let\PY@it=\textit}
\@namedef{PY@tok@gs}{\let\PY@bf=\textbf}
\@namedef{PY@tok@gp}{\let\PY@bf=\textbf\def\PY@tc##1{\textcolor[rgb]{0.00,0.00,0.50}{##1}}}
\@namedef{PY@tok@go}{\def\PY@tc##1{\textcolor[rgb]{0.44,0.44,0.44}{##1}}}
\@namedef{PY@tok@gt}{\def\PY@tc##1{\textcolor[rgb]{0.00,0.27,0.87}{##1}}}
\@namedef{PY@tok@err}{\def\PY@bc##1{{\setlength{\fboxsep}{\string -\fboxrule}\fcolorbox[rgb]{1.00,0.00,0.00}{1,1,1}{\strut ##1}}}}
\@namedef{PY@tok@kc}{\let\PY@bf=\textbf\def\PY@tc##1{\textcolor[rgb]{0.00,0.50,0.00}{##1}}}
\@namedef{PY@tok@kd}{\let\PY@bf=\textbf\def\PY@tc##1{\textcolor[rgb]{0.00,0.50,0.00}{##1}}}
\@namedef{PY@tok@kn}{\let\PY@bf=\textbf\def\PY@tc##1{\textcolor[rgb]{0.00,0.50,0.00}{##1}}}
\@namedef{PY@tok@kr}{\let\PY@bf=\textbf\def\PY@tc##1{\textcolor[rgb]{0.00,0.50,0.00}{##1}}}
\@namedef{PY@tok@bp}{\def\PY@tc##1{\textcolor[rgb]{0.00,0.50,0.00}{##1}}}
\@namedef{PY@tok@fm}{\def\PY@tc##1{\textcolor[rgb]{0.00,0.00,1.00}{##1}}}
\@namedef{PY@tok@vc}{\def\PY@tc##1{\textcolor[rgb]{0.10,0.09,0.49}{##1}}}
\@namedef{PY@tok@vg}{\def\PY@tc##1{\textcolor[rgb]{0.10,0.09,0.49}{##1}}}
\@namedef{PY@tok@vi}{\def\PY@tc##1{\textcolor[rgb]{0.10,0.09,0.49}{##1}}}
\@namedef{PY@tok@vm}{\def\PY@tc##1{\textcolor[rgb]{0.10,0.09,0.49}{##1}}}
\@namedef{PY@tok@sa}{\def\PY@tc##1{\textcolor[rgb]{0.73,0.13,0.13}{##1}}}
\@namedef{PY@tok@sb}{\def\PY@tc##1{\textcolor[rgb]{0.73,0.13,0.13}{##1}}}
\@namedef{PY@tok@sc}{\def\PY@tc##1{\textcolor[rgb]{0.73,0.13,0.13}{##1}}}
\@namedef{PY@tok@dl}{\def\PY@tc##1{\textcolor[rgb]{0.73,0.13,0.13}{##1}}}
\@namedef{PY@tok@s2}{\def\PY@tc##1{\textcolor[rgb]{0.73,0.13,0.13}{##1}}}
\@namedef{PY@tok@sh}{\def\PY@tc##1{\textcolor[rgb]{0.73,0.13,0.13}{##1}}}
\@namedef{PY@tok@s1}{\def\PY@tc##1{\textcolor[rgb]{0.73,0.13,0.13}{##1}}}
\@namedef{PY@tok@mb}{\def\PY@tc##1{\textcolor[rgb]{0.40,0.40,0.40}{##1}}}
\@namedef{PY@tok@mf}{\def\PY@tc##1{\textcolor[rgb]{0.40,0.40,0.40}{##1}}}
\@namedef{PY@tok@mh}{\def\PY@tc##1{\textcolor[rgb]{0.40,0.40,0.40}{##1}}}
\@namedef{PY@tok@mi}{\def\PY@tc##1{\textcolor[rgb]{0.40,0.40,0.40}{##1}}}
\@namedef{PY@tok@il}{\def\PY@tc##1{\textcolor[rgb]{0.40,0.40,0.40}{##1}}}
\@namedef{PY@tok@mo}{\def\PY@tc##1{\textcolor[rgb]{0.40,0.40,0.40}{##1}}}
\@namedef{PY@tok@ch}{\let\PY@it=\textit\def\PY@tc##1{\textcolor[rgb]{0.24,0.48,0.48}{##1}}}
\@namedef{PY@tok@cm}{\let\PY@it=\textit\def\PY@tc##1{\textcolor[rgb]{0.24,0.48,0.48}{##1}}}
\@namedef{PY@tok@cpf}{\let\PY@it=\textit\def\PY@tc##1{\textcolor[rgb]{0.24,0.48,0.48}{##1}}}
\@namedef{PY@tok@c1}{\let\PY@it=\textit\def\PY@tc##1{\textcolor[rgb]{0.24,0.48,0.48}{##1}}}
\@namedef{PY@tok@cs}{\let\PY@it=\textit\def\PY@tc##1{\textcolor[rgb]{0.24,0.48,0.48}{##1}}}

\def\PYZbs{\char`\\}
\def\PYZus{\char`\_}
\def\PYZob{\char`\{}
\def\PYZcb{\char`\}}
\def\PYZca{\char`\^}
\def\PYZam{\char`\&}
\def\PYZlt{\char`\<}
\def\PYZgt{\char`\>}
\def\PYZsh{\char`\#}
\def\PYZpc{\char`\%}
\def\PYZdl{\char`\$}
\def\PYZhy{\char`\-}
\def\PYZsq{\char`\'}
\def\PYZdq{\char`\"}
\def\PYZti{\char`\~}
% for compatibility with earlier versions
\def\PYZat{@}
\def\PYZlb{[}
\def\PYZrb{]}
\makeatother


    % For linebreaks inside Verbatim environment from package fancyvrb. 
    \makeatletter
        \newbox\Wrappedcontinuationbox 
        \newbox\Wrappedvisiblespacebox 
        \newcommand*\Wrappedvisiblespace {\textcolor{red}{\textvisiblespace}} 
        \newcommand*\Wrappedcontinuationsymbol {\textcolor{red}{\llap{\tiny$\m@th\hookrightarrow$}}} 
        \newcommand*\Wrappedcontinuationindent {3ex } 
        \newcommand*\Wrappedafterbreak {\kern\Wrappedcontinuationindent\copy\Wrappedcontinuationbox} 
        % Take advantage of the already applied Pygments mark-up to insert 
        % potential linebreaks for TeX processing. 
        %        {, <, #, %, $, ' and ": go to next line. 
        %        _, }, ^, &, >, - and ~: stay at end of broken line. 
        % Use of \textquotesingle for straight quote. 
        \newcommand*\Wrappedbreaksatspecials {% 
            \def\PYGZus{\discretionary{\char`\_}{\Wrappedafterbreak}{\char`\_}}% 
            \def\PYGZob{\discretionary{}{\Wrappedafterbreak\char`\{}{\char`\{}}% 
            \def\PYGZcb{\discretionary{\char`\}}{\Wrappedafterbreak}{\char`\}}}% 
            \def\PYGZca{\discretionary{\char`\^}{\Wrappedafterbreak}{\char`\^}}% 
            \def\PYGZam{\discretionary{\char`\&}{\Wrappedafterbreak}{\char`\&}}% 
            \def\PYGZlt{\discretionary{}{\Wrappedafterbreak\char`\<}{\char`\<}}% 
            \def\PYGZgt{\discretionary{\char`\>}{\Wrappedafterbreak}{\char`\>}}% 
            \def\PYGZsh{\discretionary{}{\Wrappedafterbreak\char`\#}{\char`\#}}% 
            \def\PYGZpc{\discretionary{}{\Wrappedafterbreak\char`\%}{\char`\%}}% 
            \def\PYGZdl{\discretionary{}{\Wrappedafterbreak\char`\$}{\char`\$}}% 
            \def\PYGZhy{\discretionary{\char`\-}{\Wrappedafterbreak}{\char`\-}}% 
            \def\PYGZsq{\discretionary{}{\Wrappedafterbreak\textquotesingle}{\textquotesingle}}% 
            \def\PYGZdq{\discretionary{}{\Wrappedafterbreak\char`\"}{\char`\"}}% 
            \def\PYGZti{\discretionary{\char`\~}{\Wrappedafterbreak}{\char`\~}}% 
        } 
        % Some characters . , ; ? ! / are not pygmentized. 
        % This macro makes them "active" and they will insert potential linebreaks 
        \newcommand*\Wrappedbreaksatpunct {% 
            \lccode`\~`\.\lowercase{\def~}{\discretionary{\hbox{\char`\.}}{\Wrappedafterbreak}{\hbox{\char`\.}}}% 
            \lccode`\~`\,\lowercase{\def~}{\discretionary{\hbox{\char`\,}}{\Wrappedafterbreak}{\hbox{\char`\,}}}% 
            \lccode`\~`\;\lowercase{\def~}{\discretionary{\hbox{\char`\;}}{\Wrappedafterbreak}{\hbox{\char`\;}}}% 
            \lccode`\~`\:\lowercase{\def~}{\discretionary{\hbox{\char`\:}}{\Wrappedafterbreak}{\hbox{\char`\:}}}% 
            \lccode`\~`\?\lowercase{\def~}{\discretionary{\hbox{\char`\?}}{\Wrappedafterbreak}{\hbox{\char`\?}}}% 
            \lccode`\~`\!\lowercase{\def~}{\discretionary{\hbox{\char`\!}}{\Wrappedafterbreak}{\hbox{\char`\!}}}% 
            \lccode`\~`\/\lowercase{\def~}{\discretionary{\hbox{\char`\/}}{\Wrappedafterbreak}{\hbox{\char`\/}}}% 
            \catcode`\.\active
            \catcode`\,\active 
            \catcode`\;\active
            \catcode`\:\active
            \catcode`\?\active
            \catcode`\!\active
            \catcode`\/\active 
            \lccode`\~`\~ 	
        }
    \makeatother

    \let\OriginalVerbatim=\Verbatim
    \makeatletter
    \renewcommand{\Verbatim}[1][1]{%
        %\parskip\z@skip
        \sbox\Wrappedcontinuationbox {\Wrappedcontinuationsymbol}%
        \sbox\Wrappedvisiblespacebox {\FV@SetupFont\Wrappedvisiblespace}%
        \def\FancyVerbFormatLine ##1{\hsize\linewidth
            \vtop{\raggedright\hyphenpenalty\z@\exhyphenpenalty\z@
                \doublehyphendemerits\z@\finalhyphendemerits\z@
                \strut ##1\strut}%
        }%
        % If the linebreak is at a space, the latter will be displayed as visible
        % space at end of first line, and a continuation symbol starts next line.
        % Stretch/shrink are however usually zero for typewriter font.
        \def\FV@Space {%
            \nobreak\hskip\z@ plus\fontdimen3\font minus\fontdimen4\font
            \discretionary{\copy\Wrappedvisiblespacebox}{\Wrappedafterbreak}
            {\kern\fontdimen2\font}%
        }%
        
        % Allow breaks at special characters using \PYG... macros.
        \Wrappedbreaksatspecials
        % Breaks at punctuation characters . , ; ? ! and / need catcode=\active 	
        \OriginalVerbatim[#1,codes*=\Wrappedbreaksatpunct]%
    }
    \makeatother

    % Exact colors from NB
    \definecolor{incolor}{HTML}{303F9F}
    \definecolor{outcolor}{HTML}{D84315}
    \definecolor{cellborder}{HTML}{CFCFCF}
    \definecolor{cellbackground}{HTML}{F7F7F7}
    
    % prompt
    \makeatletter
    \newcommand{\boxspacing}{\kern\kvtcb@left@rule\kern\kvtcb@boxsep}
    \makeatother
    \newcommand{\prompt}[4]{
        {\ttfamily\llap{{\color{#2}[#3]:\hspace{3pt}#4}}\vspace{-\baselineskip}}
    }
    

    
    % Prevent overflowing lines due to hard-to-break entities
    \sloppy 
    % Setup hyperref package
    \hypersetup{
      breaklinks=true,  % so long urls are correctly broken across lines
      colorlinks=true,
      urlcolor=urlcolor,
      linkcolor=linkcolor,
      citecolor=citecolor,
      }
    % Slightly bigger margins than the latex defaults
    
    \geometry{verbose,tmargin=1in,bmargin=1in,lmargin=1in,rmargin=1in}
    
    

\begin{document}
    
    \maketitle
    
    

    
    \hypertarget{ux43cux430ux448ux438ux43dux43dux43eux435-ux43eux431ux443ux447ux435ux43dux438ux435-ux432-jupyter-notebook}{%
\section{Машинное обучение в Jupyter
Notebook}\label{ux43cux430ux448ux438ux43dux43dux43eux435-ux43eux431ux443ux447ux435ux43dux438ux435-ux432-jupyter-notebook}}

    Первым делом необходимо импортировать необходимые для обучения
библиотеки

    \begin{tcolorbox}[breakable, size=fbox, boxrule=1pt, pad at break*=1mm,colback=cellbackground, colframe=cellborder]
\prompt{In}{incolor}{1}{\boxspacing}
\begin{Verbatim}[commandchars=\\\{\}]
\PY{k+kn}{import} \PY{n+nn}{numpy} \PY{k}{as} \PY{n+nn}{np} \PY{c+c1}{\PYZsh{} for matrix operations}
\PY{k+kn}{from} \PY{n+nn}{sklearn}\PY{n+nn}{.}\PY{n+nn}{neural\PYZus{}network} \PY{k+kn}{import} \PY{n}{MLPRegressor} \PY{k}{as} \PY{n}{mlp} \PY{c+c1}{\PYZsh{} main ml class}
\PY{k+kn}{from} \PY{n+nn}{matplotlib} \PY{k+kn}{import} \PY{n}{pyplot} \PY{k}{as} \PY{n}{plt} \PY{c+c1}{\PYZsh{} для построения графиков}
\PY{c+c1}{\PYZsh{} show plot inside notebook}
\PY{o}{\PYZpc{}}\PY{k}{matplotlib} inline 
\PY{k+kn}{from} \PY{n+nn}{IPython}\PY{n+nn}{.}\PY{n+nn}{display} \PY{k+kn}{import} \PY{n}{set\PYZus{}matplotlib\PYZus{}formats} \PY{c+c1}{\PYZsh{}  hi\PYZhy{}res plots}
\PY{n}{set\PYZus{}matplotlib\PYZus{}formats}\PY{p}{(}\PY{l+s+s1}{\PYZsq{}}\PY{l+s+s1}{pdf}\PY{l+s+s1}{\PYZsq{}}\PY{p}{,} \PY{l+s+s1}{\PYZsq{}}\PY{l+s+s1}{png}\PY{l+s+s1}{\PYZsq{}}\PY{p}{)}
\PY{k+kn}{from} \PY{n+nn}{joblib} \PY{k+kn}{import} \PY{n}{dump}\PY{p}{,} \PY{n}{load} \PY{c+c1}{\PYZsh{} для загрузки и выгрузки моделей}
\PY{n}{plt}\PY{o}{.}\PY{n}{rcParams}\PY{p}{[}\PY{l+s+s1}{\PYZsq{}}\PY{l+s+s1}{figure.figsize}\PY{l+s+s1}{\PYZsq{}}\PY{p}{]} \PY{o}{=} \PY{p}{[}\PY{l+m+mi}{10}\PY{p}{,} \PY{l+m+mi}{10}\PY{p}{]}
\end{Verbatim}
\end{tcolorbox}

    \begin{Verbatim}[commandchars=\\\{\}]
/tmp/ipykernel\_787542/2688301736.py:7: DeprecationWarning:
`set\_matplotlib\_formats` is deprecated since IPython 7.23, directly use
`matplotlib\_inline.backend\_inline.set\_matplotlib\_formats()`
  set\_matplotlib\_formats('pdf', 'png')
    \end{Verbatim}

    \hypertarget{ux434ux43eux431ux430ux432ux43bux435ux43dux438ux435-ux434ux430ux43dux43dux44bux445}{%
\subsection{Добавление
данных}\label{ux434ux43eux431ux430ux432ux43bux435ux43dux438ux435-ux434ux430ux43dux43dux44bux445}}

У нас имеются экспериментальные данные полнофакторного эксперимента по
измерению характеристик фотонного кристалла в зависимости от времени
распыления магнетрона и мощности на нём. В качестве входных данных
выступают два массива: мощность на магнетроне X1 и время распыления X2

Выходные параметры -- ширина запрещённой зоны, нм; длина отражённой
волны, нм; степень отражения, \%

    \begin{tcolorbox}[breakable, size=fbox, boxrule=1pt, pad at break*=1mm,colback=cellbackground, colframe=cellborder]
\prompt{In}{incolor}{2}{\boxspacing}
\begin{Verbatim}[commandchars=\\\{\}]
\PY{c+c1}{\PYZsh{}input}
\PY{n}{X0} \PY{o}{=} \PY{n}{np}\PY{o}{.}\PY{n}{ones}\PY{p}{(}\PY{l+m+mi}{4}\PY{p}{)}
\PY{n}{X1} \PY{o}{=} \PY{n}{np}\PY{o}{.}\PY{n}{array}\PY{p}{(}\PY{p}{[}\PY{l+m+mi}{150}\PY{p}{,} \PY{l+m+mi}{150}\PY{p}{,} \PY{l+m+mi}{200}\PY{p}{,} \PY{l+m+mi}{200}\PY{p}{]}\PY{p}{)}
\PY{n}{X2} \PY{o}{=} \PY{n}{np}\PY{o}{.}\PY{n}{array}\PY{p}{(}\PY{p}{[}\PY{l+m+mi}{5}\PY{p}{,} \PY{l+m+mi}{10}\PY{p}{,} \PY{l+m+mi}{5}\PY{p}{,} \PY{l+m+mi}{10}\PY{p}{]}\PY{p}{)}

\PY{c+c1}{\PYZsh{}output}
\PY{n}{Y1} \PY{o}{=} \PY{n}{np}\PY{o}{.}\PY{n}{array}\PY{p}{(}\PY{p}{[}\PY{l+m+mf}{53.5}\PY{p}{,} \PY{l+m+mf}{50.5}\PY{p}{,} \PY{l+m+mf}{56.5}\PY{p}{,} \PY{l+m+mi}{38}\PY{p}{]}\PY{p}{)}
\PY{n}{Y2} \PY{o}{=} \PY{n}{np}\PY{o}{.}\PY{n}{array}\PY{p}{(}\PY{p}{[}\PY{l+m+mf}{571.5}\PY{p}{,} \PY{l+m+mf}{556.5}\PY{p}{,} \PY{l+m+mi}{575}\PY{p}{,} \PY{l+m+mf}{532.5}\PY{p}{]}\PY{p}{)}
\PY{n}{Y3} \PY{o}{=} \PY{n}{np}\PY{o}{.}\PY{n}{array}\PY{p}{(}\PY{p}{[}\PY{l+m+mf}{18.9}\PY{p}{,} \PY{l+m+mf}{8.185}\PY{p}{,} \PY{l+m+mf}{12.05}\PY{p}{,} \PY{l+m+mf}{5.775}\PY{p}{]}\PY{p}{)}
\end{Verbatim}
\end{tcolorbox}

    \hypertarget{ux43dux43eux440ux43cux430ux43bux438ux437ux430ux446ux438ux44f-ux434ux430ux43dux43dux44bux445}{%
\subsection{Нормализация
данных}\label{ux43dux43eux440ux43cux430ux43bux438ux437ux430ux446ux438ux44f-ux434ux430ux43dux43dux44bux445}}

Для корректной работы нейросети данные неоходимо нормализровать в
пределах \(x\in[-1;1]\)

    \begin{tcolorbox}[breakable, size=fbox, boxrule=1pt, pad at break*=1mm,colback=cellbackground, colframe=cellborder]
\prompt{In}{incolor}{3}{\boxspacing}
\begin{Verbatim}[commandchars=\\\{\}]
\PY{n}{X1} \PY{o}{=} \PY{n}{np}\PY{o}{.}\PY{n}{interp}\PY{p}{(}\PY{n}{X1}\PY{p}{,} \PY{p}{(}\PY{n}{X1}\PY{o}{.}\PY{n}{min}\PY{p}{(}\PY{p}{)}\PY{p}{,} \PY{n}{X1}\PY{o}{.}\PY{n}{max}\PY{p}{(}\PY{p}{)}\PY{p}{)}\PY{p}{,} \PY{p}{(}\PY{o}{\PYZhy{}}\PY{l+m+mi}{1}\PY{p}{,} \PY{o}{+}\PY{l+m+mi}{1}\PY{p}{)}\PY{p}{)}
\PY{n}{X2} \PY{o}{=} \PY{n}{np}\PY{o}{.}\PY{n}{interp}\PY{p}{(}\PY{n}{X2}\PY{p}{,} \PY{p}{(}\PY{n}{X2}\PY{o}{.}\PY{n}{min}\PY{p}{(}\PY{p}{)}\PY{p}{,} \PY{n}{X2}\PY{o}{.}\PY{n}{max}\PY{p}{(}\PY{p}{)}\PY{p}{)}\PY{p}{,} \PY{p}{(}\PY{o}{\PYZhy{}}\PY{l+m+mi}{1}\PY{p}{,} \PY{o}{+}\PY{l+m+mi}{1}\PY{p}{)}\PY{p}{)}
\end{Verbatim}
\end{tcolorbox}

    \hypertarget{ux432ux44bux431ux43eux440-ux43eux43fux442ux438ux43cux430ux43bux44cux43dux44bux445-ux43fux430ux440ux430ux43cux435ux442ux440ux43eux432-ux434ux43bux44f-ux43eux431ux443ux447ux435ux43dux438ux44f}{%
\subsection{Выбор оптимальных параметров для
обучения}\label{ux432ux44bux431ux43eux440-ux43eux43fux442ux438ux43cux430ux43bux44cux43dux44bux445-ux43fux430ux440ux430ux43cux435ux442ux440ux43eux432-ux434ux43bux44f-ux43eux431ux443ux447ux435ux43dux438ux44f}}

При обучении необходимо учитывать множество факторов, среди них:

\begin{itemize}
    \item размерность сети;
    \item число скрытых слоёв;
    \item функция активации;
    \item оптимизатор.
\end{itemize}

\hypertarget{ux43cux435ux442ux43eux434ux44b-ux43aux43eux43dux442ux440ux43eux43bux44f}{%
\subsubsection{Методы
контроля}\label{ux43cux435ux442ux43eux434ux44b-ux43aux43eux43dux442ux440ux43eux43bux44f}}

В качестве метода контроля за качеством обучения предлагается следующий
алгоритм:

\begin{enumerate}
    \item Взять заранее рассчитанные математические модели из {\Large (вставить предыдущий пункт)};
    \item Подставить в них некие случайные значения, то есть получить отображение $\mathbb{X}\rightarrow \mathbb{Y}$;
    \item Использовать полученные массивы в качестве \textit{test}-выборки для проверки качества полученной модели.
\end{enumerate}

Создадим 4 варианта тестирований:

\begin{enumerate}
    \item \textit{X\_matmod\_test, y\_matmod\_test}, для удобства отображения предсказаний моделей с неизменными размерностями на входе и выходе;
    \item \textit{X\_matmod\_test\_12, y\_matmod\_test}, для удобства отображения предсказаний модели с двумя входными векторами и тремя выходными;
    \item \textit{X\_matmod\_test\_full, y\_matmod\_test\_full}, для валидации моделей с неизменными размерностями на входе и выходе;
    \item \textit{X\_matmod\_test\_notfull, y\_matmod\_test\_full}, для валидации модели с двумя входными векторами и тремя выходными;
\end{enumerate}

    \begin{tcolorbox}[breakable, size=fbox, boxrule=1pt, pad at break*=1mm,colback=cellbackground, colframe=cellborder]
\prompt{In}{incolor}{4}{\boxspacing}
\begin{Verbatim}[commandchars=\\\{\}]
\PY{c+c1}{\PYZsh{} создадим 3 массива размерностью 1000 (входные данные для валидации)}
\PY{n}{X0\PYZus{}matmod} \PY{o}{=} \PY{n}{np}\PY{o}{.}\PY{n}{ones}\PY{p}{(}\PY{l+m+mi}{1000}\PY{p}{)}
\PY{n}{X1\PYZus{}matmod} \PY{o}{=} \PY{n}{np}\PY{o}{.}\PY{n}{linspace}\PY{p}{(}\PY{o}{\PYZhy{}}\PY{l+m+mi}{1}\PY{p}{,}\PY{l+m+mi}{1}\PY{p}{,}\PY{l+m+mi}{1000}\PY{p}{)}
\PY{n}{X2\PYZus{}matmod} \PY{o}{=} \PY{n}{np}\PY{o}{.}\PY{n}{linspace}\PY{p}{(}\PY{o}{\PYZhy{}}\PY{l+m+mi}{1}\PY{p}{,}\PY{l+m+mi}{1}\PY{p}{,}\PY{l+m+mi}{1000}\PY{p}{)}

\PY{c+c1}{\PYZsh{} создадим массивы небольшой размерности для удобства сверки}
\PY{n}{X0\PYZus{}matmod\PYZus{}test} \PY{o}{=} \PY{n}{np}\PY{o}{.}\PY{n}{ones}\PY{p}{(}\PY{l+m+mi}{4}\PY{p}{)} \PY{c+c1}{\PYZsh{} генерация 4 единиц в массив}
\PY{n}{X1\PYZus{}matmod\PYZus{}test} \PY{o}{=} \PY{n}{np}\PY{o}{.}\PY{n}{random}\PY{o}{.}\PY{n}{choice}\PY{p}{(}\PY{n}{X1\PYZus{}matmod}\PY{p}{,}\PY{n}{size} \PY{o}{=} \PY{l+m+mi}{4}\PY{p}{,} \PY{n}{replace} \PY{o}{=} \PY{k+kc}{False}\PY{p}{)} \PY{c+c1}{\PYZsh{} генерация 4 значений без повторений}
\PY{n}{X2\PYZus{}matmod\PYZus{}test} \PY{o}{=} \PY{n}{np}\PY{o}{.}\PY{n}{random}\PY{o}{.}\PY{n}{choice}\PY{p}{(}\PY{n}{X2\PYZus{}matmod}\PY{p}{,}\PY{n}{size} \PY{o}{=} \PY{l+m+mi}{4}\PY{p}{,} \PY{n}{replace} \PY{o}{=} \PY{k+kc}{False}\PY{p}{)}
\PY{n}{X\PYZus{}matmod\PYZus{}test} \PY{o}{=} \PY{n}{np}\PY{o}{.}\PY{n}{transpose}\PY{p}{(}\PY{n}{np}\PY{o}{.}\PY{n}{array}\PY{p}{(}\PY{p}{[}\PY{n}{X0\PYZus{}matmod\PYZus{}test}\PY{p}{,} \PY{n}{X1\PYZus{}matmod\PYZus{}test}\PY{p}{,} \PY{n}{X2\PYZus{}matmod\PYZus{}test}\PY{p}{]}\PY{p}{)}\PY{p}{)} \PY{c+c1}{\PYZsh{} сборка массива и транспонирование}
\PY{n}{X\PYZus{}matmod\PYZus{}test\PYZus{}12} \PY{o}{=} \PY{n}{np}\PY{o}{.}\PY{n}{transpose}\PY{p}{(}\PY{n}{np}\PY{o}{.}\PY{n}{array}\PY{p}{(}\PY{p}{[}\PY{n}{X1\PYZus{}matmod\PYZus{}test}\PY{p}{,} \PY{n}{X2\PYZus{}matmod\PYZus{}test}\PY{p}{]}\PY{p}{)}\PY{p}{)} \PY{c+c1}{\PYZsh{} специальный массив без X0 для проверки необходимости в X0}

\PY{c+c1}{\PYZsh{} генерация выходных данных из рассчитанной модели для теста}
\PY{n}{y1\PYZus{}matmod\PYZus{}test} \PY{o}{=} \PY{l+m+mf}{49.625} \PY{o}{\PYZhy{}} \PY{l+m+mf}{2.375} \PY{o}{*} \PY{n}{X1\PYZus{}matmod\PYZus{}test} \PY{o}{\PYZhy{}} \PY{l+m+mf}{5.375} \PY{o}{*} \PY{n}{X2\PYZus{}matmod\PYZus{}test} \PY{o}{\PYZhy{}} \PY{l+m+mf}{3.875} \PY{o}{*} \PY{n}{X1\PYZus{}matmod\PYZus{}test} \PY{o}{*} \PY{n}{X2\PYZus{}matmod\PYZus{}test} \PY{c+c1}{\PYZsh{} вычисляем Y при помощи X}
\PY{n}{y2\PYZus{}matmod\PYZus{}test} \PY{o}{=} \PY{l+m+mf}{558.875} \PY{o}{\PYZhy{}} \PY{l+m+mf}{5.125} \PY{o}{*} \PY{n}{X1\PYZus{}matmod\PYZus{}test} \PY{o}{\PYZhy{}} \PY{l+m+mf}{14.375} \PY{o}{*} \PY{n}{X2\PYZus{}matmod\PYZus{}test} \PY{o}{\PYZhy{}} \PY{l+m+mf}{6.875} \PY{o}{*} \PY{n}{X1\PYZus{}matmod\PYZus{}test} \PY{o}{*} \PY{n}{X2\PYZus{}matmod\PYZus{}test}
\PY{n}{y3\PYZus{}matmod\PYZus{}test} \PY{o}{=} \PY{l+m+mf}{11.122} \PY{o}{\PYZhy{}} \PY{l+m+mf}{2.315} \PY{o}{*} \PY{n}{X1\PYZus{}matmod\PYZus{}test} \PY{o}{\PYZhy{}} \PY{l+m+mf}{4.2475} \PY{o}{*} \PY{n}{X2\PYZus{}matmod\PYZus{}test} \PY{o}{+} \PY{l+m+mf}{1.11} \PY{o}{*} \PY{n}{X1\PYZus{}matmod\PYZus{}test} \PY{o}{*} \PY{n}{X2\PYZus{}matmod\PYZus{}test}
\PY{n}{y\PYZus{}matmod\PYZus{}test} \PY{o}{=} \PY{n}{np}\PY{o}{.}\PY{n}{transpose}\PY{p}{(}\PY{n}{np}\PY{o}{.}\PY{n}{array}\PY{p}{(}\PY{p}{[}\PY{n}{y1\PYZus{}matmod\PYZus{}test}\PY{p}{,} \PY{n}{y2\PYZus{}matmod\PYZus{}test}\PY{p}{,} \PY{n}{y3\PYZus{}matmod\PYZus{}test}\PY{p}{]}\PY{p}{)}\PY{p}{)}

\PY{c+c1}{\PYZsh{} получаем на выходе три массива}
\PY{n+nb}{print}\PY{p}{(}\PY{n}{X\PYZus{}matmod\PYZus{}test}\PY{p}{)}
\PY{n+nb}{print}\PY{p}{(}\PY{n}{X\PYZus{}matmod\PYZus{}test\PYZus{}12}\PY{p}{)}
\PY{n+nb}{print}\PY{p}{(}\PY{n}{y\PYZus{}matmod\PYZus{}test}\PY{p}{)}

\PY{c+c1}{\PYZsh{} генерируем большой массив для более точной сверки}
\PY{n}{X\PYZus{}matmod\PYZus{}test\PYZus{}full} \PY{o}{=} \PY{n}{np}\PY{o}{.}\PY{n}{transpose}\PY{p}{(}\PY{n}{np}\PY{o}{.}\PY{n}{array}\PY{p}{(}\PY{p}{[}\PY{n}{X0\PYZus{}matmod}\PY{p}{,} \PY{n}{X1\PYZus{}matmod}\PY{p}{,} \PY{n}{X2\PYZus{}matmod}\PY{p}{]}\PY{p}{)}\PY{p}{)} \PY{c+c1}{\PYZsh{} сборка большого массива и транспонирование}
\PY{n}{X\PYZus{}matmod\PYZus{}test\PYZus{}notfull} \PY{o}{=} \PY{n}{np}\PY{o}{.}\PY{n}{transpose}\PY{p}{(}\PY{n}{np}\PY{o}{.}\PY{n}{array}\PY{p}{(}\PY{p}{[}\PY{n}{X1\PYZus{}matmod}\PY{p}{,} \PY{n}{X2\PYZus{}matmod}\PY{p}{]}\PY{p}{)}\PY{p}{)} \PY{c+c1}{\PYZsh{} специальный массив без X0 для проверки необходимости в X0}
\PY{n}{y1\PYZus{}matmod\PYZus{}test\PYZus{}full} \PY{o}{=} \PY{l+m+mf}{49.625} \PY{o}{\PYZhy{}} \PY{l+m+mf}{2.375} \PY{o}{*} \PY{n}{X1\PYZus{}matmod} \PY{o}{\PYZhy{}} \PY{l+m+mf}{5.375} \PY{o}{*} \PY{n}{X2\PYZus{}matmod} \PY{o}{\PYZhy{}} \PY{l+m+mf}{3.875} \PY{o}{*} \PY{n}{X1\PYZus{}matmod} \PY{o}{*} \PY{n}{X2\PYZus{}matmod} \PY{c+c1}{\PYZsh{} вычисляем Y при помощи X}
\PY{n}{y2\PYZus{}matmod\PYZus{}test\PYZus{}full} \PY{o}{=} \PY{l+m+mf}{558.875} \PY{o}{\PYZhy{}} \PY{l+m+mf}{5.125} \PY{o}{*} \PY{n}{X1\PYZus{}matmod} \PY{o}{\PYZhy{}} \PY{l+m+mf}{14.375} \PY{o}{*} \PY{n}{X2\PYZus{}matmod} \PY{o}{\PYZhy{}} \PY{l+m+mf}{6.875} \PY{o}{*} \PY{n}{X1\PYZus{}matmod} \PY{o}{*} \PY{n}{X2\PYZus{}matmod}
\PY{n}{y3\PYZus{}matmod\PYZus{}test\PYZus{}full} \PY{o}{=} \PY{l+m+mf}{11.122} \PY{o}{\PYZhy{}} \PY{l+m+mf}{2.315} \PY{o}{*} \PY{n}{X1\PYZus{}matmod} \PY{o}{\PYZhy{}} \PY{l+m+mf}{4.2475} \PY{o}{*} \PY{n}{X2\PYZus{}matmod} \PY{o}{+} \PY{l+m+mf}{1.11} \PY{o}{*} \PY{n}{X1\PYZus{}matmod} \PY{o}{*} \PY{n}{X2\PYZus{}matmod}
\PY{n}{y\PYZus{}matmod\PYZus{}test\PYZus{}full} \PY{o}{=} \PY{n}{np}\PY{o}{.}\PY{n}{transpose}\PY{p}{(}\PY{n}{np}\PY{o}{.}\PY{n}{array}\PY{p}{(}\PY{p}{[}\PY{n}{y1\PYZus{}matmod\PYZus{}test\PYZus{}full}\PY{p}{,} \PY{n}{y2\PYZus{}matmod\PYZus{}test\PYZus{}full}\PY{p}{,} \PY{n}{y3\PYZus{}matmod\PYZus{}test\PYZus{}full}\PY{p}{]}\PY{p}{)}\PY{p}{)}
\end{Verbatim}
\end{tcolorbox}

    \begin{Verbatim}[commandchars=\\\{\}]
[[ 1.          0.48348348  0.33533534]
 [ 1.         -0.20920921 -0.74174174]
 [ 1.          0.73773774 -0.36536537]
 [ 1.          0.04904905 -0.28328328]]
[[ 0.48348348  0.33533534]
 [-0.20920921 -0.74174174]
 [ 0.73773774 -0.36536537]
 [ 0.04904905 -0.28328328]]
[[ 46.04604905 550.46206417   8.7583622 ]
 [ 53.50741432 569.54287771  14.92911628]
 [ 50.88119401 562.19933497  10.66683289]
 [ 51.08499841 562.7913474   12.196274  ]]
    \end{Verbatim}

    \hypertarget{ux432ux43bux438ux44fux43dux438ux435-ux440ux430ux437ux43cux435ux440ux43dux43eux441ux442ux438-ux441ux435ux442ux438-ux43dux430-ux43aux430ux447ux435ux441ux442ux432ux43e-ux43cux43eux434ux435ux43bux438}{%
\subsection{Влияние размерности сети на качество
модели}\label{ux432ux43bux438ux44fux43dux438ux435-ux440ux430ux437ux43cux435ux440ux43dux43eux441ux442ux438-ux441ux435ux442ux438-ux43dux430-ux43aux430ux447ux435ux441ux442ux432ux43e-ux43cux43eux434ux435ux43bux438}}

Создадим 2 массива \(\mathbb{X}\): с 2 и 3 элементами соответственно.
Сделано это для потому, что нейронной сети очень сложно сделать
отображение из меньшей размерности в большую. Вектор \(X0\) является ни
чем иным, как вектором-заглушкой, не вносящим изменения в модель, но
искусственно раздувающий размерность.

    \begin{tcolorbox}[breakable, size=fbox, boxrule=1pt, pad at break*=1mm,colback=cellbackground, colframe=cellborder]
\prompt{In}{incolor}{5}{\boxspacing}
\begin{Verbatim}[commandchars=\\\{\}]
\PY{n}{X\PYZus{}012\PYZus{}train} \PY{o}{=} \PY{n}{np}\PY{o}{.}\PY{n}{array}\PY{p}{(}\PY{p}{[}\PY{n}{X0}\PY{p}{,}\PY{n}{X1}\PY{p}{,}\PY{n}{X2}\PY{p}{]}\PY{p}{)} \PY{c+c1}{\PYZsh{} вектор размерности 3}
\PY{n}{X\PYZus{}12\PYZus{}train} \PY{o}{=} \PY{n}{np}\PY{o}{.}\PY{n}{array}\PY{p}{(}\PY{p}{[}\PY{n}{X1}\PY{p}{,}\PY{n}{X2}\PY{p}{]}\PY{p}{)} \PY{c+c1}{\PYZsh{} вектор размерности 2}
\PY{n}{X\PYZus{}012\PYZus{}train} 
\end{Verbatim}
\end{tcolorbox}

            \begin{tcolorbox}[breakable, size=fbox, boxrule=.5pt, pad at break*=1mm, opacityfill=0]
\prompt{Out}{outcolor}{5}{\boxspacing}
\begin{Verbatim}[commandchars=\\\{\}]
array([[ 1.,  1.,  1.,  1.],
       [-1., -1.,  1.,  1.],
       [-1.,  1., -1.,  1.]])
\end{Verbatim}
\end{tcolorbox}
        
    Создадим 1 массив \(\mathbb{Y}\), поскольку нас интересует 3 выходных
параметра. Размерность этого вектора мы варьировать не будем

    \begin{tcolorbox}[breakable, size=fbox, boxrule=1pt, pad at break*=1mm,colback=cellbackground, colframe=cellborder]
\prompt{In}{incolor}{6}{\boxspacing}
\begin{Verbatim}[commandchars=\\\{\}]
\PY{n}{Y\PYZus{}train}\PY{o}{=}\PY{n}{np}\PY{o}{.}\PY{n}{array}\PY{p}{(}\PY{p}{[}\PY{n}{Y1}\PY{p}{,}\PY{n}{Y2}\PY{p}{,}\PY{n}{Y3}\PY{p}{]}\PY{p}{)}
\PY{n}{Y\PYZus{}train}
\end{Verbatim}
\end{tcolorbox}

            \begin{tcolorbox}[breakable, size=fbox, boxrule=.5pt, pad at break*=1mm, opacityfill=0]
\prompt{Out}{outcolor}{6}{\boxspacing}
\begin{Verbatim}[commandchars=\\\{\}]
array([[ 53.5  ,  50.5  ,  56.5  ,  38.   ],
       [571.5  , 556.5  , 575.   , 532.5  ],
       [ 18.9  ,   8.185,  12.05 ,   5.775]])
\end{Verbatim}
\end{tcolorbox}
        
    \hypertarget{ux442ux440ux430ux43dux441ux43fux43eux440ux438ux440ux43eux432ux430ux43dux438ux435-ux43cux430ux441ux441ux438ux432ux43eux432}{%
\subsubsection{Транспорирование
массивов}\label{ux442ux440ux430ux43dux441ux43fux43eux440ux438ux440ux43eux432ux430ux43dux438ux435-ux43cux430ux441ux441ux438ux432ux43eux432}}

Получившиеся выше массивы не совсем подходят для обработки. Нам
необходимо отображение комбинаций \(\mathbb{X}\) в комбинации
\(\mathbb{Y}\). Для этого транспонируем обе матрицы.

    \begin{tcolorbox}[breakable, size=fbox, boxrule=1pt, pad at break*=1mm,colback=cellbackground, colframe=cellborder]
\prompt{In}{incolor}{7}{\boxspacing}
\begin{Verbatim}[commandchars=\\\{\}]
\PY{n}{X\PYZus{}012\PYZus{}train} \PY{o}{=} \PY{n}{np}\PY{o}{.}\PY{n}{transpose}\PY{p}{(}\PY{n}{X\PYZus{}012\PYZus{}train}\PY{p}{)}
\PY{n}{X\PYZus{}12\PYZus{}train} \PY{o}{=} \PY{n}{np}\PY{o}{.}\PY{n}{transpose}\PY{p}{(}\PY{n}{X\PYZus{}12\PYZus{}train}\PY{p}{)}
\PY{n}{X\PYZus{}012\PYZus{}train}
\end{Verbatim}
\end{tcolorbox}

            \begin{tcolorbox}[breakable, size=fbox, boxrule=.5pt, pad at break*=1mm, opacityfill=0]
\prompt{Out}{outcolor}{7}{\boxspacing}
\begin{Verbatim}[commandchars=\\\{\}]
array([[ 1., -1., -1.],
       [ 1., -1.,  1.],
       [ 1.,  1., -1.],
       [ 1.,  1.,  1.]])
\end{Verbatim}
\end{tcolorbox}
        
    \begin{tcolorbox}[breakable, size=fbox, boxrule=1pt, pad at break*=1mm,colback=cellbackground, colframe=cellborder]
\prompt{In}{incolor}{8}{\boxspacing}
\begin{Verbatim}[commandchars=\\\{\}]
\PY{n}{Y\PYZus{}train} \PY{o}{=} \PY{n}{np}\PY{o}{.}\PY{n}{transpose}\PY{p}{(}\PY{n}{Y\PYZus{}train}\PY{p}{)}
\PY{n}{Y\PYZus{}train}
\end{Verbatim}
\end{tcolorbox}

            \begin{tcolorbox}[breakable, size=fbox, boxrule=.5pt, pad at break*=1mm, opacityfill=0]
\prompt{Out}{outcolor}{8}{\boxspacing}
\begin{Verbatim}[commandchars=\\\{\}]
array([[ 53.5  , 571.5  ,  18.9  ],
       [ 50.5  , 556.5  ,   8.185],
       [ 56.5  , 575.   ,  12.05 ],
       [ 38.   , 532.5  ,   5.775]])
\end{Verbatim}
\end{tcolorbox}
        
    \hypertarget{ux441ux43eux437ux434ux430ux43dux438ux435-ux430ux440ux445ux438ux442ux435ux43aux442ux443ux440ux44b-ux43cux43eux434ux435ux43bux438}{%
\subsubsection{Создание архитектуры
модели}\label{ux441ux43eux437ux434ux430ux43dux438ux435-ux430ux440ux445ux438ux442ux435ux43aux442ux443ux440ux44b-ux43cux43eux434ux435ux43bux438}}

Создадим 2 модели, отвечающие за размерности векторов. Оставим все
значения по умолчанию кроме входных параметров.

    \begin{tcolorbox}[breakable, size=fbox, boxrule=1pt, pad at break*=1mm,colback=cellbackground, colframe=cellborder]
\prompt{In}{incolor}{9}{\boxspacing}
\begin{Verbatim}[commandchars=\\\{\}]
\PY{n}{model\PYZus{}012} \PY{o}{=} \PY{n}{mlp}\PY{p}{(}\PY{n}{random\PYZus{}state}\PY{o}{=}\PY{l+m+mi}{1}\PY{p}{,}\PY{n}{max\PYZus{}iter}\PY{o}{=}\PY{l+m+mi}{1000000}\PY{p}{)}
\PY{n}{model\PYZus{}12} \PY{o}{=} \PY{n}{mlp}\PY{p}{(}\PY{n}{random\PYZus{}state}\PY{o}{=}\PY{l+m+mi}{1}\PY{p}{,}\PY{n}{max\PYZus{}iter}\PY{o}{=}\PY{l+m+mi}{1000000}\PY{p}{)}
\end{Verbatim}
\end{tcolorbox}

    Проведём обучение моделей на заранее подготовленных выборках

    \begin{tcolorbox}[breakable, size=fbox, boxrule=1pt, pad at break*=1mm,colback=cellbackground, colframe=cellborder]
\prompt{In}{incolor}{10}{\boxspacing}
\begin{Verbatim}[commandchars=\\\{\}]
\PY{n}{model\PYZus{}012}\PY{o}{.}\PY{n}{fit}\PY{p}{(}\PY{n}{X\PYZus{}012\PYZus{}train}\PY{p}{,} \PY{n}{Y\PYZus{}train}\PY{p}{)}
\PY{n}{model\PYZus{}12}\PY{o}{.}\PY{n}{fit}\PY{p}{(}\PY{n}{X\PYZus{}12\PYZus{}train}\PY{p}{,} \PY{n}{Y\PYZus{}train}\PY{p}{)}
\end{Verbatim}
\end{tcolorbox}

            \begin{tcolorbox}[breakable, size=fbox, boxrule=.5pt, pad at break*=1mm, opacityfill=0]
\prompt{Out}{outcolor}{10}{\boxspacing}
\begin{Verbatim}[commandchars=\\\{\}]
MLPRegressor(max\_iter=1000000, random\_state=1)
\end{Verbatim}
\end{tcolorbox}
        
    Проведём проверку качества работы моделей при помощи заранее
заготовленных тестов. Функция \textit{score} -- функция
\(\mathbb{R}^2\). Наилучшая возможная оценка -- 1, и она может быть
отрицательной (потому что модель может быть произвольно хуже).
Постоянная модель, которая всегда предсказывает ожидаемое значение y,
игнорируя входные характеристики, получит оценку 0. \begin{equation}
R = \left(1-\frac{u}{v}\right)
\end{equation}

\begin{itemize}
\item $u$ -- остаточная сумма квадратов. Метод оценки разницы между данными и оценочной моделью. Чем меньше разница, тем лучше оценка. По существу измеряет разброс ошибок моделирования. Другими словами, он показывает, как изменение зависимой переменной в регрессионной модели не может быть объяснено с помощью модели. Как правило, более низкая остаточная сумма квадратов указывает на то, что регрессионная модель может лучше объяснить данные, в то время как более высокая остаточная сумма квадратов указывает на то, что модель плохо объясняет данные.
\begin{equation}
    u = \sum\limits_{i=1}^{n} \left(y_i-\hat{y}_i\right)^2
\end{equation}
    \begin{itemize}
        \item $y_i$ -- действительное значение;
        \item $\hat{y}_i$ -- значение, полученное моделью.
    \end{itemize}
\item $v$ -- общая сумма квадратов. Представляет собой отклонение значений зависимой переменной от выборочного среднего значения зависимой переменной. По сути, общая сумма квадратов количественно определяет общую вариацию в выборке. Его можно определить по следующей формуле:
\begin{equation}
    v = \sum\limits_{i=1}^{n} \left(y_i-\bar{y}_i\right)^2
\end{equation}
\begin{itemize}
        \item $y_i$ -- действительное значение;
        \item $\bar{y}_i$ -- среднее значение.
    \end{itemize}
\end{itemize}

    \begin{tcolorbox}[breakable, size=fbox, boxrule=1pt, pad at break*=1mm,colback=cellbackground, colframe=cellborder]
\prompt{In}{incolor}{11}{\boxspacing}
\begin{Verbatim}[commandchars=\\\{\}]
\PY{n+nb}{print}\PY{p}{(}\PY{l+s+s2}{\PYZdq{}}\PY{l+s+s2}{Модель 012}\PY{l+s+s2}{\PYZdq{}}\PY{p}{)}
\PY{n+nb}{print}\PY{p}{(}\PY{n}{model\PYZus{}012}\PY{o}{.}\PY{n}{score}\PY{p}{(}\PY{n}{X\PYZus{}matmod\PYZus{}test\PYZus{}full}\PY{p}{,}\PY{n}{y\PYZus{}matmod\PYZus{}test\PYZus{}full}\PY{p}{)}\PY{p}{)}
\PY{n+nb}{print}\PY{p}{(}\PY{n}{model\PYZus{}012}\PY{o}{.}\PY{n}{predict}\PY{p}{(}\PY{n}{X\PYZus{}matmod\PYZus{}test}\PY{p}{)}\PY{p}{)}

\PY{n+nb}{print}\PY{p}{(}\PY{l+s+s2}{\PYZdq{}}\PY{l+s+s2}{Модель 12}\PY{l+s+s2}{\PYZdq{}}\PY{p}{)}
\PY{n+nb}{print}\PY{p}{(}\PY{n}{model\PYZus{}12}\PY{o}{.}\PY{n}{score}\PY{p}{(}\PY{n}{X\PYZus{}matmod\PYZus{}test\PYZus{}notfull}\PY{p}{,}\PY{n}{y\PYZus{}matmod\PYZus{}test\PYZus{}full}\PY{p}{)}\PY{p}{)}
\PY{n+nb}{print}\PY{p}{(}\PY{n}{model\PYZus{}12}\PY{o}{.}\PY{n}{predict}\PY{p}{(}\PY{n}{X\PYZus{}matmod\PYZus{}test\PYZus{}12}\PY{p}{)}\PY{p}{)}
\end{Verbatim}
\end{tcolorbox}

    \begin{Verbatim}[commandchars=\\\{\}]
Модель 012
0.8135462211671235
[[ 43.40571923 543.68112196   9.65241689]
 [ 50.90158838 562.20556561  15.87171834]
 [ 46.43032235 552.39883876  12.27512113]
 [ 47.8070884  554.49364847  13.29834493]]
Модель 12
-16.94187285805916
[[ 36.32865331 440.61778227   5.53561853]
 [ 48.04352244 484.47184626  13.96828786]
 [ 41.28359463 459.47902463   8.88703558]
 [ 43.18840615 465.67895223  10.44771188]]
    \end{Verbatim}

    Как видно, модель с раздутой размерностью показала себя лучше. Построим
график сходимости каждой из моделей

    \begin{tcolorbox}[breakable, size=fbox, boxrule=1pt, pad at break*=1mm,colback=cellbackground, colframe=cellborder]
\prompt{In}{incolor}{12}{\boxspacing}
\begin{Verbatim}[commandchars=\\\{\}]
\PY{n}{y\PYZus{}curve\PYZus{}012} \PY{o}{=} \PY{n}{model\PYZus{}012}\PY{o}{.}\PY{n}{loss\PYZus{}curve\PYZus{}}
\PY{n}{x\PYZus{}curve\PYZus{}012} \PY{o}{=} \PY{p}{[}\PY{n}{i} \PY{k}{for} \PY{n}{i} \PY{o+ow}{in} \PY{n+nb}{range}\PY{p}{(}\PY{l+m+mi}{1}\PY{p}{,}\PY{n+nb}{len}\PY{p}{(}\PY{n}{y\PYZus{}curve\PYZus{}012}\PY{p}{)}\PY{o}{+}\PY{l+m+mi}{1}\PY{p}{)}\PY{p}{]}
\PY{n+nb}{print}\PY{p}{(}\PY{l+s+sa}{f}\PY{l+s+s1}{\PYZsq{}}\PY{l+s+s1}{Модель 012 сошлась за }\PY{l+s+si}{\PYZob{}}\PY{n+nb}{len}\PY{p}{(}\PY{n}{x\PYZus{}curve\PYZus{}012}\PY{p}{)}\PY{l+s+si}{\PYZcb{}}\PY{l+s+s1}{ итераций, потери составили }\PY{l+s+si}{\PYZob{}}\PY{n}{model\PYZus{}012}\PY{o}{.}\PY{n}{best\PYZus{}loss\PYZus{}}\PY{l+s+si}{\PYZcb{}}\PY{l+s+s1}{\PYZsq{}}\PY{p}{)}

\PY{n}{y\PYZus{}curve\PYZus{}12} \PY{o}{=} \PY{n}{model\PYZus{}12}\PY{o}{.}\PY{n}{loss\PYZus{}curve\PYZus{}}
\PY{n}{x\PYZus{}curve\PYZus{}12} \PY{o}{=} \PY{p}{[}\PY{n}{i} \PY{k}{for} \PY{n}{i} \PY{o+ow}{in} \PY{n+nb}{range}\PY{p}{(}\PY{l+m+mi}{1}\PY{p}{,}\PY{n+nb}{len}\PY{p}{(}\PY{n}{y\PYZus{}curve\PYZus{}12}\PY{p}{)}\PY{o}{+}\PY{l+m+mi}{1}\PY{p}{)}\PY{p}{]}
\PY{n+nb}{print}\PY{p}{(}\PY{l+s+sa}{f}\PY{l+s+s1}{\PYZsq{}}\PY{l+s+s1}{Модель 12 сошлась за }\PY{l+s+si}{\PYZob{}}\PY{n+nb}{len}\PY{p}{(}\PY{n}{x\PYZus{}curve\PYZus{}12}\PY{p}{)}\PY{l+s+si}{\PYZcb{}}\PY{l+s+s1}{ итераций, потери составили }\PY{l+s+si}{\PYZob{}}\PY{n}{model\PYZus{}12}\PY{o}{.}\PY{n}{best\PYZus{}loss\PYZus{}}\PY{l+s+si}{\PYZcb{}}\PY{l+s+s1}{\PYZsq{}}\PY{p}{)}

\PY{n}{plt}\PY{o}{.}\PY{n}{plot}\PY{p}{(}\PY{n}{x\PYZus{}curve\PYZus{}12}\PY{p}{,} \PY{n}{y\PYZus{}curve\PYZus{}12}\PY{p}{,}\PY{n}{label}\PY{o}{=}\PY{l+s+s1}{\PYZsq{}}\PY{l+s+s1}{Модель 12}\PY{l+s+s1}{\PYZsq{}}\PY{p}{)}
\PY{n}{plt}\PY{o}{.}\PY{n}{plot}\PY{p}{(}\PY{n}{x\PYZus{}curve\PYZus{}012}\PY{p}{,} \PY{n}{y\PYZus{}curve\PYZus{}012}\PY{p}{,}\PY{n}{label}\PY{o}{=}\PY{l+s+s1}{\PYZsq{}}\PY{l+s+s1}{Модель 012}\PY{l+s+s1}{\PYZsq{}}\PY{p}{)}
\PY{n}{plt}\PY{o}{.}\PY{n}{xlabel}\PY{p}{(}\PY{l+s+s2}{\PYZdq{}}\PY{l+s+s2}{Число итераций}\PY{l+s+s2}{\PYZdq{}}\PY{p}{)}
\PY{n}{plt}\PY{o}{.}\PY{n}{ylabel}\PY{p}{(}\PY{l+s+s2}{\PYZdq{}}\PY{l+s+s2}{Loss}\PY{l+s+s2}{\PYZdq{}}\PY{p}{)}
\PY{n}{plt}\PY{o}{.}\PY{n}{legend}\PY{p}{(}\PY{p}{)}
\PY{n}{plt}\PY{o}{.}\PY{n}{title}\PY{p}{(}\PY{l+s+s2}{\PYZdq{}}\PY{l+s+s2}{Функция потерь двух архитектур моделей}\PY{l+s+s2}{\PYZdq{}}\PY{p}{)}
\end{Verbatim}
\end{tcolorbox}

    \begin{Verbatim}[commandchars=\\\{\}]
Модель 012 сошлась за 7864 итераций, потери составили 0.05247663242357793
Модель 12 сошлась за 4114 итераций, потери составили 0.03380976458075128
    \end{Verbatim}

            \begin{tcolorbox}[breakable, size=fbox, boxrule=.5pt, pad at break*=1mm, opacityfill=0]
\prompt{Out}{outcolor}{12}{\boxspacing}
\begin{Verbatim}[commandchars=\\\{\}]
Text(0.5, 1.0, 'Функция потерь двух архитектур моделей')
\end{Verbatim}
\end{tcolorbox}
        
    \begin{center}
    \adjustimage{max size={0.9\linewidth}{0.9\paperheight}}{output_23_2.pdf}
    \end{center}
    { \hspace*{\fill} \\}
    
    \hypertarget{ux432ux43bux438ux44fux43dux438ux435-ux447ux438ux441ux43bux430-ux441ux43aux440ux44bux442ux44bux445-ux441ux43bux43eux451ux432-ux441ux435ux442ux438-ux43dux430-ux43aux430ux447ux435ux441ux442ux432ux43e-ux43cux43eux434ux435ux43bux438}{%
\subsection{Влияние числа скрытых слоёв сети на качество
модели}\label{ux432ux43bux438ux44fux43dux438ux435-ux447ux438ux441ux43bux430-ux441ux43aux440ux44bux442ux44bux445-ux441ux43bux43eux451ux432-ux441ux435ux442ux438-ux43dux430-ux43aux430ux447ux435ux441ux442ux432ux43e-ux43cux43eux434ux435ux43bux438}}

Как мы убедились в прошлом разделе, модель должна иметь размерность на
входе не ниже размерности на выходе, иначе потери \(R^2\) слишком
высоки. Посмотрим теперь что будет, если варьировать количество скрытых
слоёв модели (например, от 1 до 100). Необходимо помнить, что слишком
сложная модель может переобучиться на небольшой выборке. Будем сохранять
модели в отдельные файлы, поскольку их обучение занимает длительное
время

    \begin{tcolorbox}[breakable, size=fbox, boxrule=1pt, pad at break*=1mm,colback=cellbackground, colframe=cellborder]
\prompt{In}{incolor}{ }{\boxspacing}
\begin{Verbatim}[commandchars=\\\{\}]
\PY{n}{loss} \PY{o}{=} \PY{p}{[}\PY{p}{]}
\PY{n}{points} \PY{o}{=} \PY{p}{[}\PY{p}{]}
\PY{n}{iters} \PY{o}{=} \PY{p}{[}\PY{p}{]}
\PY{k}{for} \PY{n}{layers} \PY{o+ow}{in} \PY{n+nb}{range}\PY{p}{(}\PY{l+m+mi}{2}\PY{p}{,}\PY{l+m+mi}{101}\PY{p}{)}\PY{p}{:}
    \PY{n}{varymodel} \PY{o}{=} \PY{n}{mlp}\PY{p}{(}\PY{n}{random\PYZus{}state}\PY{o}{=}\PY{l+m+mi}{1}\PY{p}{,}\PY{n}{max\PYZus{}iter}\PY{o}{=}\PY{l+m+mi}{1000000}\PY{p}{,}\PY{n}{hidden\PYZus{}layer\PYZus{}sizes}\PY{o}{=}\PY{p}{(}\PY{n}{layers}\PY{p}{)}\PY{p}{)}
    \PY{n}{varymodel}\PY{o}{.}\PY{n}{fit}\PY{p}{(}\PY{n}{X\PYZus{}012\PYZus{}train}\PY{p}{,} \PY{n}{Y\PYZus{}train}\PY{p}{)}
    \PY{n}{loss}\PY{o}{.}\PY{n}{append}\PY{p}{(}\PY{n}{varymodel}\PY{o}{.}\PY{n}{best\PYZus{}loss\PYZus{}}\PY{p}{)}
    \PY{n}{points}\PY{o}{.}\PY{n}{append}\PY{p}{(}\PY{n}{varymodel}\PY{o}{.}\PY{n}{loss\PYZus{}curve\PYZus{}}\PY{p}{)}
    \PY{n}{iters}\PY{o}{.}\PY{n}{append}\PY{p}{(}\PY{n}{varymodel}\PY{o}{.}\PY{n}{n\PYZus{}iter\PYZus{}}\PY{p}{)}
    \PY{c+c1}{\PYZsh{}dump(varymodel,f\PYZsq{}models/layers/\PYZob{}layers\PYZcb{}.joblib\PYZsq{})}
    \PY{n+nb}{print}\PY{p}{(}\PY{n}{layers}\PY{p}{)}
\end{Verbatim}
\end{tcolorbox}

    Для того, чтобы не проводить многократную тренировку модели, проведём
тренировку один раз, а в остальные будем загружать модели из готовых
файлов.

    \begin{tcolorbox}[breakable, size=fbox, boxrule=1pt, pad at break*=1mm,colback=cellbackground, colframe=cellborder]
\prompt{In}{incolor}{13}{\boxspacing}
\begin{Verbatim}[commandchars=\\\{\}]
\PY{n}{loss} \PY{o}{=} \PY{p}{[}\PY{p}{]}
\PY{n}{points} \PY{o}{=} \PY{p}{[}\PY{p}{]}
\PY{n}{iters} \PY{o}{=} \PY{p}{[}\PY{p}{]}
\PY{n}{scores} \PY{o}{=} \PY{p}{[}\PY{p}{]}
\PY{k}{for} \PY{n}{layers} \PY{o+ow}{in} \PY{n+nb}{range}\PY{p}{(}\PY{l+m+mi}{2}\PY{p}{,}\PY{l+m+mi}{101}\PY{p}{)}\PY{p}{:}
    \PY{n}{varymodel} \PY{o}{=} \PY{n}{load}\PY{p}{(}\PY{l+s+sa}{f}\PY{l+s+s1}{\PYZsq{}}\PY{l+s+s1}{models/layers/}\PY{l+s+si}{\PYZob{}}\PY{n}{layers}\PY{l+s+si}{\PYZcb{}}\PY{l+s+s1}{.joblib}\PY{l+s+s1}{\PYZsq{}}\PY{p}{)}
    \PY{n}{loss}\PY{o}{.}\PY{n}{append}\PY{p}{(}\PY{n}{varymodel}\PY{o}{.}\PY{n}{best\PYZus{}loss\PYZus{}}\PY{p}{)}
    \PY{n}{points}\PY{o}{.}\PY{n}{append}\PY{p}{(}\PY{n}{varymodel}\PY{o}{.}\PY{n}{loss\PYZus{}curve\PYZus{}}\PY{p}{)}
    \PY{n}{iters}\PY{o}{.}\PY{n}{append}\PY{p}{(}\PY{n}{varymodel}\PY{o}{.}\PY{n}{n\PYZus{}iter\PYZus{}}\PY{p}{)}
    \PY{n}{scores}\PY{o}{.}\PY{n}{append}\PY{p}{(}\PY{n}{varymodel}\PY{o}{.}\PY{n}{score}\PY{p}{(}\PY{n}{X\PYZus{}matmod\PYZus{}test\PYZus{}full}\PY{p}{,}\PY{n}{y\PYZus{}matmod\PYZus{}test\PYZus{}full}\PY{p}{)}\PY{p}{)}
\end{Verbatim}
\end{tcolorbox}

    \begin{tcolorbox}[breakable, size=fbox, boxrule=1pt, pad at break*=1mm,colback=cellbackground, colframe=cellborder]
\prompt{In}{incolor}{14}{\boxspacing}
\begin{Verbatim}[commandchars=\\\{\}]
\PY{n}{loss}\PY{p}{[}\PY{l+m+mi}{5}\PY{p}{]} \PY{o}{=} \PY{l+m+mi}{0} \PY{c+c1}{\PYZsh{} выборочная ошибка, занулим её}
\PY{n}{scores}\PY{p}{[}\PY{l+m+mi}{5}\PY{p}{]} \PY{o}{=} \PY{l+m+mi}{0}
\end{Verbatim}
\end{tcolorbox}

    \begin{tcolorbox}[breakable, size=fbox, boxrule=1pt, pad at break*=1mm,colback=cellbackground, colframe=cellborder]
\prompt{In}{incolor}{15}{\boxspacing}
\begin{Verbatim}[commandchars=\\\{\}]
\PY{n}{layers} \PY{o}{=} \PY{n+nb}{list}\PY{p}{(}\PY{n+nb}{range}\PY{p}{(}\PY{l+m+mi}{2}\PY{p}{,}\PY{l+m+mi}{101}\PY{p}{)}\PY{p}{)}
\PY{n}{plt}\PY{o}{.}\PY{n}{plot}\PY{p}{(}\PY{n}{layers}\PY{p}{,} \PY{n}{loss}\PY{p}{)}
\PY{n}{plt}\PY{o}{.}\PY{n}{xlabel}\PY{p}{(}\PY{l+s+s2}{\PYZdq{}}\PY{l+s+s2}{Число слоёв}\PY{l+s+s2}{\PYZdq{}}\PY{p}{)}
\PY{n}{plt}\PY{o}{.}\PY{n}{ylabel}\PY{p}{(}\PY{l+s+s2}{\PYZdq{}}\PY{l+s+s2}{Loss}\PY{l+s+s2}{\PYZdq{}}\PY{p}{)}
\PY{n}{plt}\PY{o}{.}\PY{n}{title}\PY{p}{(}\PY{l+s+s2}{\PYZdq{}}\PY{l+s+s2}{Влияние числа слоёв на потери при обучении}\PY{l+s+s2}{\PYZdq{}}\PY{p}{)}
\end{Verbatim}
\end{tcolorbox}

            \begin{tcolorbox}[breakable, size=fbox, boxrule=.5pt, pad at break*=1mm, opacityfill=0]
\prompt{Out}{outcolor}{15}{\boxspacing}
\begin{Verbatim}[commandchars=\\\{\}]
Text(0.5, 1.0, 'Влияние числа слоёв на потери при обучении')
\end{Verbatim}
\end{tcolorbox}
        
    \begin{center}
    \adjustimage{max size={0.9\linewidth}{0.9\paperheight}}{output_29_1.pdf}
    \end{center}
    { \hspace*{\fill} \\}
    
    \begin{tcolorbox}[breakable, size=fbox, boxrule=1pt, pad at break*=1mm,colback=cellbackground, colframe=cellborder]
\prompt{In}{incolor}{16}{\boxspacing}
\begin{Verbatim}[commandchars=\\\{\}]
\PY{k}{for} \PY{n}{i} \PY{o+ow}{in} \PY{n+nb}{range} \PY{p}{(}\PY{n+nb}{len}\PY{p}{(}\PY{n}{iters}\PY{p}{)}\PY{p}{)}\PY{p}{:}
    \PY{c+c1}{\PYZsh{}print(i)}
    \PY{k}{if} \PY{p}{(}\PY{n}{i}\PY{o}{\PYZgt{}}\PY{l+m+mi}{10} \PY{o+ow}{and} \PY{n}{i} \PY{o}{\PYZpc{}} \PY{l+m+mi}{5} \PY{o}{!=} \PY{l+m+mi}{0}\PY{p}{)}\PY{p}{:}
        \PY{k}{continue}
    \PY{n}{x} \PY{o}{=} \PY{n+nb}{list}\PY{p}{(}\PY{n+nb}{range}\PY{p}{(}\PY{l+m+mi}{1}\PY{p}{,}\PY{n}{iters}\PY{p}{[}\PY{n}{i}\PY{p}{]}\PY{o}{+}\PY{l+m+mi}{1}\PY{p}{)}\PY{p}{)}
    \PY{n}{y} \PY{o}{=} \PY{n}{points}\PY{p}{[}\PY{n}{i}\PY{p}{]}   
    \PY{n}{plt}\PY{o}{.}\PY{n}{plot}\PY{p}{(}\PY{n}{x}\PY{p}{,} \PY{n}{y}\PY{p}{,} \PY{n}{label}\PY{o}{=}\PY{l+s+sa}{f}\PY{l+s+s1}{\PYZsq{}}\PY{l+s+s1}{Слой }\PY{l+s+si}{\PYZob{}}\PY{n}{i}\PY{o}{+}\PY{l+m+mi}{2}\PY{l+s+si}{\PYZcb{}}\PY{l+s+s1}{\PYZsq{}}\PY{p}{)}
\PY{n}{plt}\PY{o}{.}\PY{n}{xlabel}\PY{p}{(}\PY{l+s+s2}{\PYZdq{}}\PY{l+s+s2}{Число итераций}\PY{l+s+s2}{\PYZdq{}}\PY{p}{)}
\PY{n}{plt}\PY{o}{.}\PY{n}{ylabel}\PY{p}{(}\PY{l+s+s2}{\PYZdq{}}\PY{l+s+s2}{Потери}\PY{l+s+s2}{\PYZdq{}}\PY{p}{)}
\PY{n}{plt}\PY{o}{.}\PY{n}{title}\PY{p}{(}\PY{l+s+s2}{\PYZdq{}}\PY{l+s+s2}{Влияние числа слоёв на качество модели}\PY{l+s+s2}{\PYZdq{}}\PY{p}{)}
\PY{n}{plt}\PY{o}{.}\PY{n}{legend}\PY{p}{(}\PY{p}{)}
\end{Verbatim}
\end{tcolorbox}

            \begin{tcolorbox}[breakable, size=fbox, boxrule=.5pt, pad at break*=1mm, opacityfill=0]
\prompt{Out}{outcolor}{16}{\boxspacing}
\begin{Verbatim}[commandchars=\\\{\}]
<matplotlib.legend.Legend at 0x7fe3db519d90>
\end{Verbatim}
\end{tcolorbox}
        
    \begin{center}
    \adjustimage{max size={0.9\linewidth}{0.9\paperheight}}{output_30_1.pdf}
    \end{center}
    { \hspace*{\fill} \\}
    
    Самым главным параметром при обучении будет являться её оценка. Проведём
оценку модели

    \begin{tcolorbox}[breakable, size=fbox, boxrule=1pt, pad at break*=1mm,colback=cellbackground, colframe=cellborder]
\prompt{In}{incolor}{17}{\boxspacing}
\begin{Verbatim}[commandchars=\\\{\}]
\PY{n}{plt}\PY{o}{.}\PY{n}{plot}\PY{p}{(}\PY{n+nb}{list}\PY{p}{(}\PY{n+nb}{range}\PY{p}{(}\PY{l+m+mi}{2}\PY{p}{,}\PY{l+m+mi}{101}\PY{p}{)}\PY{p}{)}\PY{p}{,} \PY{n}{scores}\PY{p}{)}
\PY{n}{plt}\PY{o}{.}\PY{n}{xlabel}\PY{p}{(}\PY{l+s+s2}{\PYZdq{}}\PY{l+s+s2}{Число слоёв}\PY{l+s+s2}{\PYZdq{}}\PY{p}{)}
\PY{n}{plt}\PY{o}{.}\PY{n}{ylabel}\PY{p}{(}\PY{l+s+s2}{\PYZdq{}}\PY{l+s+s2}{Оценка модели}\PY{l+s+s2}{\PYZdq{}}\PY{p}{)}
\PY{n}{plt}\PY{o}{.}\PY{n}{title}\PY{p}{(}\PY{l+s+s2}{\PYZdq{}}\PY{l+s+s2}{Влияние числа слоёв на оценку модели}\PY{l+s+s2}{\PYZdq{}}\PY{p}{)}
\end{Verbatim}
\end{tcolorbox}

            \begin{tcolorbox}[breakable, size=fbox, boxrule=.5pt, pad at break*=1mm, opacityfill=0]
\prompt{Out}{outcolor}{17}{\boxspacing}
\begin{Verbatim}[commandchars=\\\{\}]
Text(0.5, 1.0, 'Влияние числа слоёв на оценку модели')
\end{Verbatim}
\end{tcolorbox}
        
    \begin{center}
    \adjustimage{max size={0.9\linewidth}{0.9\paperheight}}{output_32_1.pdf}
    \end{center}
    { \hspace*{\fill} \\}
    
    Очевидно, что нас интересует всё до десяти слоёв, дальше модель начинает
переобучаться (резкое падение оценки -- признак переобучения)

    \begin{tcolorbox}[breakable, size=fbox, boxrule=1pt, pad at break*=1mm,colback=cellbackground, colframe=cellborder]
\prompt{In}{incolor}{18}{\boxspacing}
\begin{Verbatim}[commandchars=\\\{\}]
\PY{n}{plt}\PY{o}{.}\PY{n}{plot}\PY{p}{(}\PY{n+nb}{list}\PY{p}{(}\PY{n+nb}{range}\PY{p}{(}\PY{l+m+mi}{2}\PY{p}{,}\PY{l+m+mi}{10}\PY{p}{)}\PY{p}{)}\PY{p}{,} \PY{n}{scores}\PY{p}{[}\PY{p}{:}\PY{l+m+mi}{8}\PY{p}{]}\PY{p}{)}
\PY{n}{plt}\PY{o}{.}\PY{n}{xlabel}\PY{p}{(}\PY{l+s+s2}{\PYZdq{}}\PY{l+s+s2}{Число слоёв}\PY{l+s+s2}{\PYZdq{}}\PY{p}{)}
\PY{n}{plt}\PY{o}{.}\PY{n}{ylabel}\PY{p}{(}\PY{l+s+s2}{\PYZdq{}}\PY{l+s+s2}{Оценка модели}\PY{l+s+s2}{\PYZdq{}}\PY{p}{)}
\PY{n}{plt}\PY{o}{.}\PY{n}{title}\PY{p}{(}\PY{l+s+s2}{\PYZdq{}}\PY{l+s+s2}{Влияние числа слоёв на оценку модели}\PY{l+s+s2}{\PYZdq{}}\PY{p}{)}
\PY{k}{for} \PY{n}{i} \PY{o+ow}{in} \PY{n+nb}{range}\PY{p}{(}\PY{n+nb}{len}\PY{p}{(}\PY{n}{scores}\PY{p}{[}\PY{p}{:}\PY{l+m+mi}{8}\PY{p}{]}\PY{p}{)}\PY{p}{)}\PY{p}{:}
    \PY{n+nb}{print}\PY{p}{(}\PY{l+s+sa}{f}\PY{l+s+s1}{\PYZsq{}}\PY{l+s+s1}{Число слоёв }\PY{l+s+si}{\PYZob{}}\PY{n}{i}\PY{o}{+}\PY{l+m+mi}{2}\PY{l+s+si}{\PYZcb{}}\PY{l+s+s1}{, оценка модели }\PY{l+s+si}{\PYZob{}}\PY{n}{scores}\PY{p}{[}\PY{n}{i}\PY{p}{]}\PY{l+s+si}{\PYZcb{}}\PY{l+s+s1}{\PYZsq{}}\PY{p}{)}
\end{Verbatim}
\end{tcolorbox}

    \begin{Verbatim}[commandchars=\\\{\}]
Число слоёв 2, оценка модели 0.4639444511330419
Число слоёв 3, оценка модели 0.9100311255101875
Число слоёв 4, оценка модели 0.9265799685143974
Число слоёв 5, оценка модели 0.8241257764470432
Число слоёв 6, оценка модели 0.924471640369974
Число слоёв 7, оценка модели 0
Число слоёв 8, оценка модели 0.9244552323152719
Число слоёв 9, оценка модели 0.8396396016693698
    \end{Verbatim}

    \begin{center}
    \adjustimage{max size={0.9\linewidth}{0.9\paperheight}}{output_34_1.pdf}
    \end{center}
    { \hspace*{\fill} \\}
    
    Из графика видно, что лучше всего нам подходит модель с числом слоёв,
равным 8 (далее наблюдается переобучение)

    \hypertarget{ux432ux43bux438ux44fux43dux438ux435-ux444ux443ux43dux43aux446ux438ux438-ux430ux43aux442ux438ux432ux430ux446ux438ux438-ux43dux430-ux440ux435ux437ux443ux43bux44cux442ux430ux442ux44b-ux43cux43eux434ux435ux43bux438}{%
\subsection{Влияние функции активации на результаты
модели}\label{ux432ux43bux438ux44fux43dux438ux435-ux444ux443ux43dux43aux446ux438ux438-ux430ux43aux442ux438ux432ux430ux446ux438ux438-ux43dux430-ux440ux435ux437ux443ux43bux44cux442ux430ux442ux44b-ux43cux43eux434ux435ux43bux438}}

Библиотека \textit{scikit-learn} предоставляет доступ к 4 функциям
активации:

\begin{enumerate}
    \item \textit{identity} -- функция тождества. Безоперационная активация, полезная для реализации линейного узкого места
        \begin{equation}
            f(x) = x
        \end{equation}
    \item \textit{logistic} -- сигмоидальная функция
        \begin{equation}
            f(x) = \frac{1}{1+e^{-x}}
        \end{equation}
    \item \textit{tanh} -- гиперболический тангенс
        \begin{equation}
            f(x) = \tanh = \frac{e^{x}-e^{-x}}{e^{x}+e^{-x}}
        \end{equation}
    \item \textit{relu} -- линейный выпрямитель
        \begin{equation}
            f(x) = 
            \begin{cases}
                0, x<0\\
                x, x\geqslant 0
            \end{cases}
        \end{equation}
\end{enumerate}

Обучим 4 модели при помощи каждой из функций активаций

    \begin{tcolorbox}[breakable, size=fbox, boxrule=1pt, pad at break*=1mm,colback=cellbackground, colframe=cellborder]
\prompt{In}{incolor}{19}{\boxspacing}
\begin{Verbatim}[commandchars=\\\{\}]
\PY{n}{scores\PYZus{}activation} \PY{o}{=} \PY{p}{[}\PY{p}{]}
\PY{n}{model\PYZus{}identity} \PY{o}{=} \PY{n}{mlp}\PY{p}{(}\PY{n}{random\PYZus{}state}\PY{o}{=}\PY{l+m+mi}{1}\PY{p}{,}\PY{n}{max\PYZus{}iter}\PY{o}{=}\PY{l+m+mi}{2000000}\PY{p}{,}\PY{n}{hidden\PYZus{}layer\PYZus{}sizes}\PY{o}{=}\PY{p}{(}\PY{l+m+mi}{8}\PY{p}{)}\PY{p}{,}\PYZbs{}
                     \PY{n}{activation}\PY{o}{=}\PY{l+s+s1}{\PYZsq{}}\PY{l+s+s1}{identity}\PY{l+s+s1}{\PYZsq{}}\PY{p}{)}
\PY{n}{model\PYZus{}identity}\PY{o}{.}\PY{n}{fit}\PY{p}{(}\PY{n}{X\PYZus{}012\PYZus{}train}\PY{p}{,} \PY{n}{Y\PYZus{}train}\PY{p}{)}
\PY{n}{scores\PYZus{}activation}\PY{o}{.}\PY{n}{append}\PY{p}{(}\PY{n}{model\PYZus{}identity}\PY{o}{.}\PY{n}{score}\PY{p}{(}\PY{n}{X\PYZus{}matmod\PYZus{}test\PYZus{}full}\PY{p}{,}\PY{n}{y\PYZus{}matmod\PYZus{}test\PYZus{}full}\PY{p}{)}\PY{p}{)}

\PY{n}{model\PYZus{}logistic} \PY{o}{=} \PY{n}{mlp}\PY{p}{(}\PY{n}{random\PYZus{}state}\PY{o}{=}\PY{l+m+mi}{1}\PY{p}{,}\PY{n}{max\PYZus{}iter}\PY{o}{=}\PY{l+m+mi}{2000000}\PY{p}{,}\PY{n}{hidden\PYZus{}layer\PYZus{}sizes}\PY{o}{=}\PY{p}{(}\PY{l+m+mi}{8}\PY{p}{)}\PY{p}{,}\PYZbs{}
                     \PY{n}{activation}\PY{o}{=}\PY{l+s+s1}{\PYZsq{}}\PY{l+s+s1}{logistic}\PY{l+s+s1}{\PYZsq{}}\PY{p}{)}
\PY{n}{model\PYZus{}logistic}\PY{o}{.}\PY{n}{fit}\PY{p}{(}\PY{n}{X\PYZus{}012\PYZus{}train}\PY{p}{,} \PY{n}{Y\PYZus{}train}\PY{p}{)}
\PY{n}{scores\PYZus{}activation}\PY{o}{.}\PY{n}{append}\PY{p}{(}\PY{n}{model\PYZus{}logistic}\PY{o}{.}\PY{n}{score}\PY{p}{(}\PY{n}{X\PYZus{}matmod\PYZus{}test\PYZus{}full}\PY{p}{,}\PY{n}{y\PYZus{}matmod\PYZus{}test\PYZus{}full}\PY{p}{)}\PY{p}{)}

\PY{n}{model\PYZus{}tanh} \PY{o}{=} \PY{n}{mlp}\PY{p}{(}\PY{n}{random\PYZus{}state}\PY{o}{=}\PY{l+m+mi}{1}\PY{p}{,}\PY{n}{max\PYZus{}iter}\PY{o}{=}\PY{l+m+mi}{2000000}\PY{p}{,}\PY{n}{hidden\PYZus{}layer\PYZus{}sizes}\PY{o}{=}\PY{p}{(}\PY{l+m+mi}{8}\PY{p}{)}\PY{p}{,}\PYZbs{}
                 \PY{n}{activation}\PY{o}{=}\PY{l+s+s1}{\PYZsq{}}\PY{l+s+s1}{tanh}\PY{l+s+s1}{\PYZsq{}}\PY{p}{)}
\PY{n}{model\PYZus{}tanh}\PY{o}{.}\PY{n}{fit}\PY{p}{(}\PY{n}{X\PYZus{}012\PYZus{}train}\PY{p}{,} \PY{n}{Y\PYZus{}train}\PY{p}{)}
\PY{n}{scores\PYZus{}activation}\PY{o}{.}\PY{n}{append}\PY{p}{(}\PY{n}{model\PYZus{}tanh}\PY{o}{.}\PY{n}{score}\PY{p}{(}\PY{n}{X\PYZus{}matmod\PYZus{}test\PYZus{}full}\PY{p}{,}\PY{n}{y\PYZus{}matmod\PYZus{}test\PYZus{}full}\PY{p}{)}\PY{p}{)}

\PY{n}{model\PYZus{}relu} \PY{o}{=} \PY{n}{mlp}\PY{p}{(}\PY{n}{random\PYZus{}state}\PY{o}{=}\PY{l+m+mi}{1}\PY{p}{,}\PY{n}{max\PYZus{}iter}\PY{o}{=}\PY{l+m+mi}{2000000}\PY{p}{,}\PY{n}{hidden\PYZus{}layer\PYZus{}sizes}\PY{o}{=}\PY{p}{(}\PY{l+m+mi}{8}\PY{p}{)}\PY{p}{,}\PYZbs{}
                 \PY{n}{activation}\PY{o}{=}\PY{l+s+s1}{\PYZsq{}}\PY{l+s+s1}{relu}\PY{l+s+s1}{\PYZsq{}}\PY{p}{)}
\PY{n}{model\PYZus{}relu}\PY{o}{.}\PY{n}{fit}\PY{p}{(}\PY{n}{X\PYZus{}012\PYZus{}train}\PY{p}{,} \PY{n}{Y\PYZus{}train}\PY{p}{)}
\PY{n}{scores\PYZus{}activation}\PY{o}{.}\PY{n}{append}\PY{p}{(}\PY{n}{model\PYZus{}relu}\PY{o}{.}\PY{n}{score}\PY{p}{(}\PY{n}{X\PYZus{}matmod\PYZus{}test\PYZus{}full}\PY{p}{,}\PY{n}{y\PYZus{}matmod\PYZus{}test\PYZus{}full}\PY{p}{)}\PY{p}{)}
\end{Verbatim}
\end{tcolorbox}

    \begin{tcolorbox}[breakable, size=fbox, boxrule=1pt, pad at break*=1mm,colback=cellbackground, colframe=cellborder]
\prompt{In}{incolor}{20}{\boxspacing}
\begin{Verbatim}[commandchars=\\\{\}]
\PY{n}{x} \PY{o}{=} \PY{p}{[}\PY{l+m+mi}{1}\PY{p}{,} \PY{l+m+mi}{2}\PY{p}{,} \PY{l+m+mi}{3}\PY{p}{,} \PY{l+m+mi}{4}\PY{p}{]}
\PY{n}{y} \PY{o}{=} \PY{n}{scores\PYZus{}activation}
\PY{n}{my\PYZus{}xticks} \PY{o}{=} \PY{p}{[}\PY{l+s+s1}{\PYZsq{}}\PY{l+s+s1}{identity}\PY{l+s+s1}{\PYZsq{}}\PY{p}{,}\PY{l+s+s1}{\PYZsq{}}\PY{l+s+s1}{logistic}\PY{l+s+s1}{\PYZsq{}}\PY{p}{,}\PY{l+s+s1}{\PYZsq{}}\PY{l+s+s1}{tanh}\PY{l+s+s1}{\PYZsq{}}\PY{p}{,}\PY{l+s+s1}{\PYZsq{}}\PY{l+s+s1}{relu}\PY{l+s+s1}{\PYZsq{}}\PY{p}{]}
\PY{n}{plt}\PY{o}{.}\PY{n}{xticks}\PY{p}{(}\PY{n}{x}\PY{p}{,} \PY{n}{my\PYZus{}xticks}\PY{p}{)}
\PY{n}{plt}\PY{o}{.}\PY{n}{scatter}\PY{p}{(}\PY{n}{x}\PY{p}{,}\PY{n}{y}\PY{p}{)}
\PY{n}{plt}\PY{o}{.}\PY{n}{xlabel}\PY{p}{(}\PY{l+s+s2}{\PYZdq{}}\PY{l+s+s2}{Способ обучения}\PY{l+s+s2}{\PYZdq{}}\PY{p}{)}
\PY{n}{plt}\PY{o}{.}\PY{n}{ylabel}\PY{p}{(}\PY{l+s+s2}{\PYZdq{}}\PY{l+s+s2}{Оценка модели}\PY{l+s+s2}{\PYZdq{}}\PY{p}{)}
\PY{n}{plt}\PY{o}{.}\PY{n}{title}\PY{p}{(}\PY{l+s+s2}{\PYZdq{}}\PY{l+s+s2}{Влияние числа слоёв на оценку модели}\PY{l+s+s2}{\PYZdq{}}\PY{p}{)}
\PY{n+nb}{print}\PY{p}{(}\PY{n}{scores\PYZus{}activation}\PY{p}{)}
\end{Verbatim}
\end{tcolorbox}

    \begin{Verbatim}[commandchars=\\\{\}]
[0.9275591949671171, -0.03702885533324888, -0.03702338056124536,
0.9244552323152719]
    \end{Verbatim}

    \begin{center}
    \adjustimage{max size={0.9\linewidth}{0.9\paperheight}}{output_38_1.pdf}
    \end{center}
    { \hspace*{\fill} \\}
    
    Из графика видно, что наилучший результат показывает отсутствие функции
активации

    \hypertarget{ux432ux43bux438ux44fux43dux438ux435-ux441ux43eux43bux432ux435ux440ux430-ux43dux430-ux440ux435ux437ux443ux43bux44cux442ux430ux442ux44b-ux43cux43eux434ux435ux43bux438}{%
\subsection{Влияние солвера на результаты
модели}\label{ux432ux43bux438ux44fux43dux438ux435-ux441ux43eux43bux432ux435ux440ux430-ux43dux430-ux440ux435ux437ux443ux43bux44cux442ux430ux442ux44b-ux43cux43eux434ux435ux43bux438}}

Под солвером понимается один из алгоритмов для обучения нейросети.
Библиотека \textit{scikit-learn} предоставляет доступ к 3 алгоритмам
обучения:

\begin{enumerate}
    \item \textit{lbfgs} -- алгоритм Бройдена - Флетчера - Гольдфарба - Шанно, рекомендуется для небольших датасетов;
    \item \textit{sgd} -- стохастический градиентный спуск, один из классических методов обучения; 
    \item \textit{adam} -- метод адаптивной скорости обучения, используется по умолчанию.
\end{enumerate}

Обучим модель с использованием этих алгоритмов

    \begin{tcolorbox}[breakable, size=fbox, boxrule=1pt, pad at break*=1mm,colback=cellbackground, colframe=cellborder]
\prompt{In}{incolor}{21}{\boxspacing}
\begin{Verbatim}[commandchars=\\\{\}]
\PY{n}{algorithms} \PY{o}{=} \PY{p}{[}\PY{p}{]}
\PY{n}{model\PYZus{}lbfgs} \PY{o}{=} \PY{n}{mlp}\PY{p}{(}\PY{n}{random\PYZus{}state}\PY{o}{=}\PY{l+m+mi}{1}\PY{p}{,}\PY{n}{max\PYZus{}iter}\PY{o}{=}\PY{l+m+mi}{2000000}\PY{p}{,}\PY{n}{hidden\PYZus{}layer\PYZus{}sizes}\PY{o}{=}\PY{p}{(}\PY{l+m+mi}{8}\PY{p}{)}\PY{p}{,}\PYZbs{}
                  \PY{n}{activation}\PY{o}{=}\PY{l+s+s1}{\PYZsq{}}\PY{l+s+s1}{identity}\PY{l+s+s1}{\PYZsq{}}\PY{p}{,}\PY{n}{solver}\PY{o}{=}\PY{l+s+s1}{\PYZsq{}}\PY{l+s+s1}{lbfgs}\PY{l+s+s1}{\PYZsq{}}\PY{p}{)}
\PY{n}{model\PYZus{}lbfgs}\PY{o}{.}\PY{n}{fit}\PY{p}{(}\PY{n}{X\PYZus{}012\PYZus{}train}\PY{p}{,} \PY{n}{Y\PYZus{}train}\PY{p}{)}
\PY{n}{algorithms}\PY{o}{.}\PY{n}{append}\PY{p}{(}\PY{n}{model\PYZus{}lbfgs}\PY{o}{.}\PY{n}{score}\PY{p}{(}\PY{n}{X\PYZus{}matmod\PYZus{}test\PYZus{}full}\PY{p}{,}\PY{n}{y\PYZus{}matmod\PYZus{}test\PYZus{}full}\PY{p}{)}\PY{p}{)}

\PY{n}{model\PYZus{}adam} \PY{o}{=} \PY{n}{mlp}\PY{p}{(}\PY{n}{random\PYZus{}state}\PY{o}{=}\PY{l+m+mi}{1}\PY{p}{,}\PY{n}{max\PYZus{}iter}\PY{o}{=}\PY{l+m+mi}{2000000}\PY{p}{,}\PY{n}{hidden\PYZus{}layer\PYZus{}sizes}\PY{o}{=}\PY{p}{(}\PY{l+m+mi}{8}\PY{p}{)}\PY{p}{,}\PYZbs{}
                 \PY{n}{activation}\PY{o}{=}\PY{l+s+s1}{\PYZsq{}}\PY{l+s+s1}{identity}\PY{l+s+s1}{\PYZsq{}}\PY{p}{,}\PY{n}{solver}\PY{o}{=}\PY{l+s+s1}{\PYZsq{}}\PY{l+s+s1}{adam}\PY{l+s+s1}{\PYZsq{}}\PY{p}{)}
\PY{n}{model\PYZus{}adam}\PY{o}{.}\PY{n}{fit}\PY{p}{(}\PY{n}{X\PYZus{}012\PYZus{}train}\PY{p}{,} \PY{n}{Y\PYZus{}train}\PY{p}{)}
\PY{n}{algorithms}\PY{o}{.}\PY{n}{append}\PY{p}{(}\PY{n}{model\PYZus{}adam}\PY{o}{.}\PY{n}{score}\PY{p}{(}\PY{n}{X\PYZus{}matmod\PYZus{}test\PYZus{}full}\PY{p}{,}\PY{n}{y\PYZus{}matmod\PYZus{}test\PYZus{}full}\PY{p}{)}\PY{p}{)}
\end{Verbatim}
\end{tcolorbox}

    При обучении модели методом стохастического градиентного спуска
нейросеть не смогла добиться оптимальных параметров и обучение было
аварийно завершено

    \begin{tcolorbox}[breakable, size=fbox, boxrule=1pt, pad at break*=1mm,colback=cellbackground, colframe=cellborder]
\prompt{In}{incolor}{22}{\boxspacing}
\begin{Verbatim}[commandchars=\\\{\}]
\PY{n}{x} \PY{o}{=} \PY{p}{[}\PY{l+m+mi}{1}\PY{p}{,} \PY{l+m+mi}{2}\PY{p}{]}
\PY{n}{y} \PY{o}{=} \PY{n}{algorithms}
\PY{n}{my\PYZus{}xticks} \PY{o}{=} \PY{p}{[}\PY{l+s+s1}{\PYZsq{}}\PY{l+s+s1}{lbfgs}\PY{l+s+s1}{\PYZsq{}}\PY{p}{,}\PY{l+s+s1}{\PYZsq{}}\PY{l+s+s1}{adam}\PY{l+s+s1}{\PYZsq{}}\PY{p}{]}
\PY{n}{plt}\PY{o}{.}\PY{n}{xticks}\PY{p}{(}\PY{n}{x}\PY{p}{,} \PY{n}{my\PYZus{}xticks}\PY{p}{)}
\PY{n}{plt}\PY{o}{.}\PY{n}{scatter}\PY{p}{(}\PY{n}{x}\PY{p}{,}\PY{n}{y}\PY{p}{)}
\PY{n}{plt}\PY{o}{.}\PY{n}{xlabel}\PY{p}{(}\PY{l+s+s2}{\PYZdq{}}\PY{l+s+s2}{Оптимизационный алгоритм}\PY{l+s+s2}{\PYZdq{}}\PY{p}{)}
\PY{n}{plt}\PY{o}{.}\PY{n}{ylabel}\PY{p}{(}\PY{l+s+s2}{\PYZdq{}}\PY{l+s+s2}{Оценка модели}\PY{l+s+s2}{\PYZdq{}}\PY{p}{)}
\PY{n}{plt}\PY{o}{.}\PY{n}{title}\PY{p}{(}\PY{l+s+s2}{\PYZdq{}}\PY{l+s+s2}{Влияние оптимизационного алгоритма на оценку модели}\PY{l+s+s2}{\PYZdq{}}\PY{p}{)}
\PY{n+nb}{print}\PY{p}{(}\PY{n}{algorithms}\PY{p}{)}
\end{Verbatim}
\end{tcolorbox}

    \begin{Verbatim}[commandchars=\\\{\}]
[0.9247783928754306, 0.9275591949671171]
    \end{Verbatim}

    \begin{center}
    \adjustimage{max size={0.9\linewidth}{0.9\paperheight}}{output_43_1.pdf}
    \end{center}
    { \hspace*{\fill} \\}
    
    \hypertarget{ux43fux440ux43eux432ux435ux440ux43aux430-ux43fux43eux43bux443ux447ux435ux43dux43dux43eux439-ux43cux43eux434ux435ux43bux438}{%
\subsection{Проверка полученной
модели}\label{ux43fux440ux43eux432ux435ux440ux43aux430-ux43fux43eux43bux443ux447ux435ux43dux43dux43eux439-ux43cux43eux434ux435ux43bux438}}

После проведения экспериментов с моделями, попробуем проверить
полученную модель на всей экспериментальной выборке. Обучим верную
модель

    \begin{tcolorbox}[breakable, size=fbox, boxrule=1pt, pad at break*=1mm,colback=cellbackground, colframe=cellborder]
\prompt{In}{incolor}{23}{\boxspacing}
\begin{Verbatim}[commandchars=\\\{\}]
\PY{n}{final\PYZus{}model} \PY{o}{=} \PY{n}{mlp}\PY{p}{(}\PY{n}{random\PYZus{}state}\PY{o}{=}\PY{l+m+mi}{1}\PY{p}{,}\PY{n}{max\PYZus{}iter}\PY{o}{=}\PY{l+m+mi}{2000000}\PY{p}{,}\PY{n}{hidden\PYZus{}layer\PYZus{}sizes}\PY{o}{=}\PY{p}{(}\PY{l+m+mi}{8}\PY{p}{)}\PY{p}{,}\PYZbs{}
                  \PY{n}{activation}\PY{o}{=}\PY{l+s+s1}{\PYZsq{}}\PY{l+s+s1}{identity}\PY{l+s+s1}{\PYZsq{}}\PY{p}{,}\PY{n}{solver}\PY{o}{=}\PY{l+s+s1}{\PYZsq{}}\PY{l+s+s1}{adam}\PY{l+s+s1}{\PYZsq{}}\PY{p}{)}
\PY{n}{final\PYZus{}model}\PY{o}{.}\PY{n}{fit}\PY{p}{(}\PY{n}{X\PYZus{}012\PYZus{}train}\PY{p}{,} \PY{n}{Y\PYZus{}train}\PY{p}{)}
\end{Verbatim}
\end{tcolorbox}

            \begin{tcolorbox}[breakable, size=fbox, boxrule=.5pt, pad at break*=1mm, opacityfill=0]
\prompt{Out}{outcolor}{23}{\boxspacing}
\begin{Verbatim}[commandchars=\\\{\}]
MLPRegressor(activation='identity', hidden\_layer\_sizes=8, max\_iter=2000000,
             random\_state=1)
\end{Verbatim}
\end{tcolorbox}
        
    \begin{tcolorbox}[breakable, size=fbox, boxrule=1pt, pad at break*=1mm,colback=cellbackground, colframe=cellborder]
\prompt{In}{incolor}{24}{\boxspacing}
\begin{Verbatim}[commandchars=\\\{\}]
\PY{n}{final\PYZus{}model}\PY{o}{.}\PY{n}{score}\PY{p}{(}\PY{n}{X\PYZus{}matmod\PYZus{}test\PYZus{}full}\PY{p}{,}\PY{n}{y\PYZus{}matmod\PYZus{}test\PYZus{}full}\PY{p}{)}
\end{Verbatim}
\end{tcolorbox}

            \begin{tcolorbox}[breakable, size=fbox, boxrule=.5pt, pad at break*=1mm, opacityfill=0]
\prompt{Out}{outcolor}{24}{\boxspacing}
\begin{Verbatim}[commandchars=\\\{\}]
0.9275591949671171
\end{Verbatim}
\end{tcolorbox}
        
    Наша модель набрала 0.92 из 1, что является хорошим результатом при
обучении на 4 входных данных

    \hypertarget{ux43fux43eux43bux443ux447ux435ux43dux438ux435-ux440ux435ux437ux443ux43bux44cux442ux430ux442ux43eux432-ux438ux437-ux43cux43eux434ux435ux43bux438}{%
\subsection{Получение результатов из
модели}\label{ux43fux43eux43bux443ux447ux435ux43dux438ux435-ux440ux435ux437ux443ux43bux44cux442ux430ux442ux43eux432-ux438ux437-ux43cux43eux434ux435ux43bux438}}

После обучения модели встаёт вопрос: как получать из неё результат. Для
этого в библиотеке \textit{scikit-learn} в классе \textit{MLPRegressor}
метод \textit{predict}. Попробуем получить из модели какое-нибудь
предсказание, например, для точки \(X=[1, 0, 0]\), что соответствует
времени напыления \(t=7.5\) минут при мощности \(P=175\) Вт

    \begin{tcolorbox}[breakable, size=fbox, boxrule=1pt, pad at break*=1mm,colback=cellbackground, colframe=cellborder]
\prompt{In}{incolor}{25}{\boxspacing}
\begin{Verbatim}[commandchars=\\\{\}]
\PY{p}{[}\PY{p}{[}\PY{n}{a}\PY{p}{,}\PY{n}{b}\PY{p}{,}\PY{n}{c}\PY{p}{]}\PY{p}{]} \PY{o}{=} \PY{n}{final\PYZus{}model}\PY{o}{.}\PY{n}{predict}\PY{p}{(}\PY{p}{[}\PY{p}{[}\PY{l+m+mi}{1}\PY{p}{,}\PY{l+m+mi}{0}\PY{p}{,}\PY{l+m+mi}{0}\PY{p}{]}\PY{p}{]}\PY{p}{)} 
\PY{n+nb}{print}\PY{p}{(}\PY{l+s+sa}{f}\PY{l+s+s1}{\PYZsq{}\PYZsq{}\PYZsq{}}\PY{l+s+s1}{Данные по модели}
\PY{l+s+s1}{Ширина запрещённой зоны }\PY{l+s+si}{\PYZob{}}\PY{n+nb}{round}\PY{p}{(}\PY{n}{a}\PY{p}{,}\PY{l+m+mi}{2}\PY{p}{)}\PY{l+s+si}{\PYZcb{}}\PY{l+s+s1}{ нм}
\PY{l+s+s1}{Длина отражённой волны }\PY{l+s+si}{\PYZob{}}\PY{n+nb}{round}\PY{p}{(}\PY{n}{b}\PY{p}{,}\PY{l+m+mi}{2}\PY{p}{)}\PY{l+s+si}{\PYZcb{}}\PY{l+s+s1}{ нм}
\PY{l+s+s1}{Процент отражённого света }\PY{l+s+si}{\PYZob{}}\PY{n+nb}{round}\PY{p}{(}\PY{n}{c}\PY{p}{,}\PY{l+m+mi}{2}\PY{p}{)}\PY{l+s+si}{\PYZcb{}}\PY{l+s+s1}{ \PYZpc{}}\PY{l+s+s1}{\PYZsq{}\PYZsq{}\PYZsq{}}\PY{p}{)}
\end{Verbatim}
\end{tcolorbox}

    \begin{Verbatim}[commandchars=\\\{\}]
Данные по модели
Ширина запрещённой зоны 49.63 нм
Длина отражённой волны 558.62 нм
Процент отражённого света 11.23 \%
    \end{Verbatim}

    Реальные данные (вычисленные по модели):

\begin{itemize}
\item Ширина запрещённой зоны 49.625 нм
\item Длина отражённой волны 558.875 нм
\item Процент отражённого света 11.122 \%
\end{itemize}

Отклонения составили:

\begin{equation}
\delta S = \frac{S_\text{real}-S_\text{model}}{S_\text{real}}\cdot 100\%=\frac{49.625-49.63}{49.625}\cdot 100\%=−0.01\%
\end{equation}

\begin{equation}
\delta\lambda = \frac{\lambda_\text{real}-\lambda_\text{model}}{\lambda_\text{real}}\cdot 100\%=\frac{558.675-558.62}{558.75}\cdot 100\%=0.0098\%
\end{equation}

\begin{equation}
\delta R = \frac{R_\text{real}-R_\text{model}}{R_\text{real}}\cdot 100\%=\frac{11.122-11.23}{11.122}\cdot 100\%=−0.97\%
\end{equation}

Отклонения от матмодели получились меньше одного процента

    \begin{tcolorbox}[breakable, size=fbox, boxrule=1pt, pad at break*=1mm,colback=cellbackground, colframe=cellborder]
\prompt{In}{incolor}{26}{\boxspacing}
\begin{Verbatim}[commandchars=\\\{\}]
\PY{n+nb}{print}\PY{p}{(}\PY{l+s+sa}{f}\PY{l+s+s1}{\PYZsq{}}\PY{l+s+s1}{Предсказание модели }\PY{l+s+si}{\PYZob{}}\PY{n}{final\PYZus{}model}\PY{o}{.}\PY{n}{predict}\PY{p}{(}\PY{n}{X\PYZus{}matmod\PYZus{}test}\PY{p}{)}\PY{l+s+si}{\PYZcb{}}\PY{l+s+s1}{\PYZsq{}}\PY{p}{)}
\PY{n+nb}{print}\PY{p}{(}\PY{l+s+sa}{f}\PY{l+s+s1}{\PYZsq{}}\PY{l+s+s1}{Реальные значения }\PY{l+s+si}{\PYZob{}}\PY{n}{y\PYZus{}matmod\PYZus{}test}\PY{l+s+si}{\PYZcb{}}\PY{l+s+s1}{\PYZsq{}}\PY{p}{)}
\end{Verbatim}
\end{tcolorbox}

    \begin{Verbatim}[commandchars=\\\{\}]
Предсказание модели [[ 46.67454687 551.32429405   8.68390614]
 [ 54.10855985 570.35738101  14.86237024]
 [ 49.83670087 560.09384672  11.07154185]
 [ 51.03116084 562.44344424  12.31720136]]
Реальные значения [[ 46.04604905 550.46206417   8.7583622 ]
 [ 53.50741432 569.54287771  14.92911628]
 [ 50.88119401 562.19933497  10.66683289]
 [ 51.08499841 562.7913474   12.196274  ]]
    \end{Verbatim}

    Сохраним полученную модель

    \begin{tcolorbox}[breakable, size=fbox, boxrule=1pt, pad at break*=1mm,colback=cellbackground, colframe=cellborder]
\prompt{In}{incolor}{27}{\boxspacing}
\begin{Verbatim}[commandchars=\\\{\}]
\PY{n}{dump}\PY{p}{(}\PY{n}{final\PYZus{}model}\PY{p}{,}\PY{l+s+sa}{f}\PY{l+s+s1}{\PYZsq{}}\PY{l+s+s1}{models/final\PYZus{}model.joblib}\PY{l+s+s1}{\PYZsq{}}\PY{p}{)}
\end{Verbatim}
\end{tcolorbox}

            \begin{tcolorbox}[breakable, size=fbox, boxrule=.5pt, pad at break*=1mm, opacityfill=0]
\prompt{Out}{outcolor}{27}{\boxspacing}
\begin{Verbatim}[commandchars=\\\{\}]
['models/final\_model.joblib']
\end{Verbatim}
\end{tcolorbox}
        

    % Add a bibliography block to the postdoc
    
    
    
\end{document}
